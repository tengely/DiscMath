% !TEX root = lectnote.tex
% !TEX spellcheck = en_GB-oed 


\chapter{Pascal's triangle}\label{cha:Pascal}

%In this Chapter we introduce Pascal's triangle, 
%then investigate some of its properties. 
%This Chapter relies heavily on the notion of binomial coefficients introduced in Section~\ref{sec:nchoosek}, 
%and the reader is advised to recall the contents of Section~\ref{sec:nchoosek} before continuing. 

Let us create a triangle from numbers in the following way. 
Let us write 1 to the top. 
This we call row zero of the triangle. 
Then every row of the triangle contains one more numbers than the row before, 
aligned in a way that every number is lower left and/or lower right from the numbers in the row above. 
We start and end every row by 1, 
and in between we write numbers which are the sums of the two numbers above them, 
that is, we write the sum of the upper left and upper right numbers. 
Thus in the first row (right below the top 1) 
we write 1 to lower left and to lower right of this number. 
Then in the second row we write 1, 2, 1, 
such that 2 is in between the two 1's of the first row. 
In the third row, we write 1, 3, 3, 1, etc. (see Table~\ref{tab:PascalTriangle}). 
This way, one can easily compute the numbers occurring in the triangle row after row. 
This triangle is called Pascal's triangle, 
named after the French polymath Blaise Pascal (1623--1662).%\footnote{
%Some historical remarks of Pascal???
%}

\begin{table}[!htb]
\caption{Pascal's triangle.}\label{tab:PascalTriangle}
\begin{center}
\begin{tabular}{cccccccccccccc} 
%$n=0$:
& & & & & & 1\\
\noalign{\smallskip\smallskip} 
%$n=1$:
& & & & & 1 & & 1\\
\noalign{\smallskip\smallskip} 
%$n=2$:
& & & & 1 & & 2 & & 1\\
\noalign{\smallskip\smallskip} 
%$n=3$:
& & & 1 & & 3 & & 3 & & 1\\
\noalign{\smallskip\smallskip} 
%$n=4$:
& & 1 & & 4 & & 6 & & 4 & & 1\\
\noalign{\smallskip\smallskip} 
& 1 & & 5 & & 10 & & 10 & & 5 & & 1 \\
\noalign{\smallskip\smallskip} 
1 & & 6 & & 15 & & 20 & & 15 & & 6 & & 1 \\
\noalign{\smallskip\smallskip} 
\end{tabular}
\end{center}
\end{table}


Let us now take a closer look to these numbers. 
Consider for example the sixth row: 1, 6, 15, 20, 15, 6, 1. 
They look like the binomial coefficients $\binom{6}{k}$. 
Indeed, 
$1 = \binom{6}{0}$, 
$6 = \binom{6}{1}$, 
$15 = \binom{6}{2}$, 
$20 = \binom{6}{3}$, 
$15 = \binom{6}{4}$, 
$6 = \binom{6}{5}$, 
$1 = \binom{6}{6}$. 
It seems that (at least for this small part of the triangle), 
in the $n$th row the binomial coefficients $\binom{n}{k}$ occur for $k=0, 1, 2, \dots , n$. 
This is true for the first row ($\binom{1}{0}=1$, $\binom{1}{1}=1$), 
and even for the zero row: $\binom{0}{0}=1$. 
That is, it looks like Pascal's triangle is the same as the triangle of the binomial coefficients, 
where in the $n$th row we write the binomial coefficients 
$\binom{n}{0}$, $\binom{n}{1}$, $\binom{n}{2}$, $\dots $, $\binom{n}{n}$ such that we align the midpoints of the rows (Table~\ref{tab:PascalTriangleBinomial}). 


\begin{table}[!htb]
\caption{Triangle of Binomial coefficients.}\label{tab:PascalTriangleBinomial}
\begin{center}
\begin{tabular}{cccccccccccccc} 
%$n=0$:
& & & & & & $\binom{0}{0}$ \\
\noalign{\smallskip\smallskip} 
%$n=1$:
& & & & & $\binom{1}{0}$ & & $\binom{1}{1}$ \\
\noalign{\smallskip\smallskip} 
%$n=2$:
& & & & $\binom{2}{0}$ & & $\binom{2}{1}$ & & $\binom{2}{2}$ \\
\noalign{\smallskip\smallskip} 
%$n=3$:
& & & $\binom{3}{0}$ & & $\binom{3}{1}$ & & $\binom{3}{2}$ & & $\binom{3}{3}$ \\
\noalign{\smallskip\smallskip} 
%$n=4$:
& & $\binom{4}{0}$ & & $\binom{4}{1}$ & & $\binom{4}{2}$ & & $\binom{4}{3}$ & & $\binom{4}{4}$ \\
\noalign{\smallskip\smallskip} 
& $\binom{5}{0}$ & & $\binom{5}{1}$ & & $\binom{5}{2}$ & & $\binom{5}{3}$ & & $\binom{5}{4}$ & & $\binom{5}{5}$ \\
\noalign{\smallskip\smallskip} 
$\binom{6}{0}$ & & $\binom{6}{1}$ & & $\binom{6}{2}$ & & $\binom{6}{3}$ & & $\binom{6}{4}$ & & $\binom{6}{5}$ & & $\binom{6}{6}$ \\
\noalign{\smallskip\smallskip} 
\end{tabular}
\end{center}
\end{table}

How can we prove that the two triangles are one and the same? 
One way to do it would be to prove that they can be generated by the same rule. 
Pascal's triangle was generated such that every row starts and ends with 1, 
and every other number is the sum of the two numbers right above it. 
Considering the $n$th row in Table~\ref{tab:PascalTriangleBinomial}, 
it starts by $\binom{n}{0}=1$, and it ends with $\binom{n}{n} = 1$. 
Thus we only need to check whether every other number is the sum of the two numbers above it. 
The $k$th number in the $n$th row is $\binom{n}{k}$ (every row starts with the zeroth number), 
the two numbers above it are the $(k-1)$st and $k$th of row $n-1$, 
that is, $\binom{n-1}{k-1}$ and $\binom{n-1}{k}$. 
Thus, if we prove that $\binom{n}{k} = \binom{n-1}{k-1} + \binom{n-1}{k}$, 
then the two triangles are indeed the same. 

\begin{proposition}\label{prop:binomsum}
For positive integers $k\leq n$ we have 
\[
\binom{n}{k} = \binom{n-1}{k-1} + \binom{n-1}{k}.
\]
\end{proposition}

\begin{proof}
Let us substitute the formula \eqref{eq:binom} into the right-hand side:
\begin{align*}
&{ } \binom{n-1}{k-1} + \binom{n-1}{k} \\
&= \frac{(n-1)!}{(k-1)! \cdot (n-1-(k-1))!} + \frac{(n-1)!}{k! \cdot (n-1-k)!} \\
&= \frac{(n-1)!}{(k-1)! \cdot (n-k)!} + \frac{(n-1)!}{k! \cdot (n-k-1)!} \\
&= \frac{(n-1)!\cdot k + (n-1)! \cdot (n-k)}{k! \cdot (n-k)!} = \frac{(n-1)!\cdot (k + n-k)}{k! \cdot (n-k)!}\\
&= \frac{(n-1)!\cdot n}{k! \cdot (n-k)!} = \frac{n!}{k! \cdot (n-k)!} = \binom{n}{k}.
\end{align*}
\end{proof}

\begin{exercise}\label{ex:pascalbinomialsame}
Create a precise proof using induction that the two triangles are the same. 
\end{exercise}

This proof is a correct one, 
but not necessarily satisfying. 
It contains calculations, 
but does not show the reason \emph{why} the sum of the binomial coefficients 
$\binom{n-1}{k-1}$ and $\binom{n-1}{k}$ is really $\binom{n}{k}$. 
One might wonder if there is an ``easier'' proof, 
which only uses the definition of $\binom{n}{k}$. 
Indeed there is, as we show now. 

\begin{proof}[Second proof of Proposition~\ref{prop:binomsum}.]
Let $A=\halmaz{1, 2, \dots , n}$, 
and we count the number of $k$-element subsets of $A$ in two different ways. 
On the one hand, we know that the number of $k$-element subsets of $A$ is $\binom{n}{k}$. 
On the other hand, we count the $k$-element subsets such that we first count those which contain the element $n$, 
then we count those, which do not. 

Count the number of $k$-element subsets of $A$ containing $n$ first. 
If a $k$-element subset $S$ of $A$ contains $n$, then it contains $k-1$ more elements from $\halmaz{1, 2, \dots , n-1}$. 
Such a subset can be chosen in $\binom{n-1}{k-1}$-many ways.
Thus, $A$ has $\binom{n-1}{k-1}$-many $k$-element subsets containing the element $n$. 

Now, count the number of $k$-element subsets of $A$ not containing $n$. 
If a $k$-element subset $S$ of $A$ does not contain $n$, then it contains $k$ elements from $\halmaz{1, 2, \dots , n-1}$. 
Such a subset can be chosen in $\binom{n-1}{k}$-many ways.
Thus, $A$ has $\binom{n-1}{k}$-many $k$-element subsets not containing the element $n$. 
As a $k$-element subset either contains or does not contain the element $n$, 
the number of $k$-element subsets of $A$ is $\binom{n-1}{k-1} + \binom{n-1}{k}$, 
which therefore must be the same number as $\binom{n}{k}$. 
\end{proof}

\begin{exercise}\label{ex:pascal12}
Compute the first twelve rows of Pascal's triangle. 
\end{exercise}


\section{Binomial theorem}

In the Algebra course we have learned to expand the expression $(x+y)^2$ to $x^2 + 2xy + y^2$. 
In this Section we expand $(x+y)^n$ for arbitrary nonnegative integers $n$. 
For this, let us first recall how such an expression should be calculated. 

Consider first $(x+y)^2$. 
This is the product of $(x+y)$ by itself, 
that is, $(x+y)^2 = (x+y)\cdot (x+y)$, 
and has to be computed by multiplying every term of the first factor by every term of the second factor: 
\begin{equation}\label{eq:(x+y)^2}
(x+y)^2 = (x+y) \cdot (x+y) = x^2 + xy + yx + y^2 = x^2 + 2xy + y^2. 
\end{equation}
Now, consider $(x+y)^3$. 
This is the threefold product of $(x+y)$ with itself, 
that is, $(x+y)^3  = (x+y)\cdot (x+y) \cdot (x+y)$. 
Note, that this is the same as the product $(x+y)^2 \cdot (x+y)$, 
for which the first factor we have already computed in \eqref{eq:(x+y)^2}. 
\begin{align}
\notag (x+y)^3 &= (x+y)^2 \cdot (x+y) = (x^2 + 2xy + y^2)\cdot (x+y) \\
\notag &= x^3 + x^2y + 2x^2y + 2xy^2 + xy^2 + y^3 \\
\notag &= x^3 + 3x^2y + 3 xy^2 + y^3. 
\end{align}
This way, we can easily continue calculating the higher powers of $(x+y)$:
\begin{align}
\notag (x+y)^4 &= (x+y)^3 \cdot (x+y) = (x^3 + 3x^2y + 3 xy^2 + y^3)\cdot (x+y) \\
\notag &= x^4 + x^3y + 3x^3y + 3x^2y^2 + 3x^2y^2 + 3xy^3 + xy^3 + y^4 \\
\notag &= x^4 + 4x^3y + 6x^2y^2 + 4xy^3 + y^4. \\
\notag (x+y)^5 &= (x+y)^4 \cdot (x+y) = (x^4 + 4x^3y + 6x^2y^2 + 4xy^3 + y^4)\cdot (x+y) \\
\notag &= x^5 + x^4y + 4x^4y + 4x^3y^2 + 6x^3y^2 + 6x^2y^3 + 4x^2y^3 + 4xy^4 + xy^4 + y^5 \\
\notag &= x^5 + 5x^4y + 10x^3y^2 + 10x^2y^3 + 5xy^4 + y^5. \\
\notag (x+y)^6 &= (x+y)^5 \cdot (x+y) = (x^5 + 5x^4y + 10x^3y^2 + 10x^2y^3 + 5xy^4 + y^5)\cdot (x+y) \\
\notag &= x^6 + x^5y + 5x^5y + 5x^4y^2 + 10x^4y^2 + 10x^3y^3 + 10x^3y^3 + 10x^2y^4 \\
\notag &+ 5x^2y^4 + 5xy^5 + xy^5 + y^6 \\
\notag &= x^6 + 6x^5y + 15x^4y^2 + 20x^3y^3 + 15x^2y^4 + 6xy^5 + y^6. 
\end{align}
Let us summarize our findings: % and complete our list with $(x+y)^0$ and $(x+y)^1$: 
\begin{align*}
(x+y)^2 & = x^2 + 2xy + y^2, \\
(x+y)^3 &= x^3 + 3x^2y + 3 xy^2 + y^3, \\
(x+y)^4 &= x^4 + 4x^3y + 6x^2y^2 + 4xy^3 + y^4, \\
(x+y)^5 &= x^5 + 5x^4y + 10x^3y^2 + 10x^2y^3 + 5xy^4 + y^5, \\
(x+y)^6 &= x^6 + 6x^5y + 15x^4y^2 + 20x^3y^3 + 15x^2y^4 + 6xy^5 + y^6. 
\end{align*}

Now, wait a minute! 
The coefficients arising in these expressions are exactly the numbers occurring in Pascal's triangle. 
%what we calculated in Exercise~\ref{ex:smallnchoosek}.
Indeed, the coefficients of $(x+y)^6$ are
$1 = \binom{6}{0}$, 
$6 = \binom{6}{1}$, 
$15 = \binom{6}{2}$, 
$20 = \binom{6}{3}$, 
$15 = \binom{6}{4}$, 
$6 = \binom{6}{5}$, 
$1 = \binom{6}{6}$. 
This cannot be a coincidence! 
It looks like that when we expand $(x+y)^n$, 
then the coefficient for the term $x^{n-k}y^k$ is $\binom{n}{k}$. 
This is always the case, not only for the first six powers. 
This is the statement of the binomial theorem. 

\begin{theorem}[Binomial theorem]\label{thm:binomial}
Let $n$ be a natural number. 
Then 
\begin{align*}
(x+y)^n &= x^n + n x^{n-1}y + \binom{n}{2} x^{n-2}y^2 + \dots + \binom{n}{n-2} x^2 y^{n-2} + n x y^{n-1} + y^n \\
&= \sum_{k=0}^n \binom{n}{k} x^{n-k}y^k. 
\end{align*}
\end{theorem}

\begin{proof}%[Induction proof]
Note first, that the Binomial theorem holds for $n=0$ and $n=1$, as well: 
$(x+y)^0 = 1 = \binom{0}{0} x^0 y^0$, 
$(x+y)^1 = x + y = \binom{1}{0} x^1 y^0 + \binom{1}{1} x^0 y^1$. 
Now, we can prove the theorem by induction on $n$. 
Assume that the statement holds for $n-1$, 
that is, 
\[
(x+y)^{n-1} = \sum_{k=0}^{n-1} \binom{n-1}{k} x^{n-1-k}y^k. 
\]
This is the induction hypothesis. 
Now, compute $(x+y)^n$ using the same method as before, 
and use the induction hypothesis for expanding $(x+y)^{n-1}$: 
\begin{align}
\notag (x+y)^n &= (x+y)^{n-1} \cdot (x+y) = \left( \sum_{k=0}^{n-1} \binom{n-1}{k} x^{n-1-k}y^k \right) \cdot \left( x + y \right) \\
\notag &= \sum_{k=0}^{n-1} \binom{n-1}{k} x^{n-1-k}y^k \cdot x + \sum_{k=0}^{n-1} \binom{n-1}{k} x^{n-1-k}y^k \cdot y \\
\label{eq:x^nfront} &= \sum_{k=0}^{n-1} \binom{n-1}{k} x^{n-k}y^k + \sum_{k=0}^{n-1} \binom{n-1}{k} x^{n-1-k}y^{k+1} \\
\label{eq:binomreindex} &= x^n + \sum_{k=1}^{n-1} \binom{n-1}{k} x^{n-k}y^k + \sum_{k=0}^{n-2} \binom{n-1}{k} x^{n-1-k}y^{k+1} + y^n \\
\notag &= x^n + \sum_{k=1}^{n-1} \binom{n-1}{k} x^{n-k}y^k + \sum_{k=1}^{n-1} \binom{n-1}{k-1} x^{n-k}y^k + y^n \\
\label{eq:binomsum} &= x^n + \sum_{k=1}^{n-1} \left( \binom{n-1}{k} + \binom{n-1}{k-1}\right) x^{n-k}y^k + y^n \\
\notag &= x^n + \sum_{k=1}^{n-1} \binom{n}{k} x^{n-k}y^k + y^n = \sum_{k=0}^n \binom{n}{k} x^{n-k}y^k. 
\end{align}
Here, we have separated $x^n$ and $y^n$ from the sums in \eqref{eq:x^nfront}, 
then ``re-indexed'' the second sum in \eqref{eq:binomreindex} to find the coefficient of the common terms $x^{n-k}y^k$ (for $k = 1, 2, \dots , n-1$) of the two sums. 
Finally, in \eqref{eq:binomsum} we used the generating rule of Pascal's triangle (Proposition~\ref{prop:binomsum}). 
\end{proof}

\begin{exercise}\label{ex:binomialwriteout}
Repeat the proof by ``writing out'' all sums. 
\end{exercise}

Now we understand why binomial coefficients are called like that: 
because they arise as the coefficients in the $n$th power of binomial sums. 
Moreover, 
the proof of the Binomial theorem revealed that raising $(x+y)$ to the next power 
affects the coefficients exactly the same way as we generate Pascal's triangle. 
Nevertheless, 
one can find another argument, which explains ``better'' why the binomial coefficients arise in the $n$th power. 

Consider $(x+y)^6$:
\[
(x+y)^6 = x^6 + 6x^5y + 15x^4y^2 + 20x^3y^3 + 15x^2y^4 + 6xy^5 + y^6. 
\] 
How do we obtain the coefficient 15 for $x^4y^2$? 
Now, $(x+y)^6$ is the 6-fold product of $(x+y)$ by itself: 
\[
(x+y)^6 = (x+y) \cdot (x+y) \cdot (x+y) \cdot (x+y) \cdot (x+y) \cdot (x+y). 
\]
The multiplication of these six factors is carried out by choosing a term from each factor ($x$ or $y$) in every possible way, 
multiplying these six terms, and then adding the resulting products together. 
Thus the coefficient of $x^4y^2$ is the number of possibilities to choose four times the $x$ and two times the $y$ out of the six factors. 
Altogether there are six $y$'s to choose from, 
and we need to choose two of them (and the remaining four factors will be chosen as $x$). 
This can be done in $\binom{6}{2} = 15$-many ways. 
%Now, we need to determine which 2 of the 6 y's we choose. 
%This can be done in $\binom{6}{2}$-many ways, 
Therefore the coefficient of $x^4y^2$ is $\binom{6}{2} = 15$. 

\begin{exercise}\label{ex:binomialnchoosek}
Prove the Binomial Theorem using the argument provided above. 
\end{exercise}

The Binomial theorem can be used to calculate several $n$th powers. 
For example, 
choosing $y=1$, every power of $y$ is 1, as well, thus 
\begin{align*}
(x+1)^n &= x^n + n x^{n-1}\cdot 1 + \binom{n}{2} x^{n-2}\cdot 1^2 + \dots + n x \cdot 1^{n-1} + 1^n \\
&= x^n + n x^{n-1} + \binom{n}{2} x^{n-2} + \dots + n x + 1 = \sum_{k=0}^n \binom{n}{k} x^k. 
\end{align*}

\begin{exercise}\label{ex:binomial1+1}
Write $x=y=1$ into the Binomial theorem. 
Note that this provides a second proof for Proposition~\ref{prop:sumofbinomial}. 
\end{exercise}

Alternatively, we can substitute $-y$ instead of $y$ in the Binomial theorem, obtaining 
\begin{align*}
(x-y)^n &= x^n + n x^{n-1}\cdot (-y) + \binom{n}{2} x^{n-2}\cdot (-y)^2 + \dots + n x \cdot (-y)^{n-1} + (-y)^n \\
&= x^n - n x^{n-1} y + \binom{n}{2} x^{n-2} y^2 - \dots + (-1)^{n-1} n x y^{n-1} + (-1)^n y^n \\
&= \sum_{k=0}^n (-1)^k \binom{n}{k} x^{n-k}y^{k}. 
\end{align*}

Choosing $y=-1$ yields 
\begin{align*}
(x-1)^n &= x^n + n x^{n-1}\cdot (-1) + \binom{n}{2} x^{n-2}\cdot (-1)^2 + \dots + n x \cdot (-1)^{n-1} + (-1)^n \\
&= x^n - n x^{n-1} + \binom{n}{2} x^{n-2} - \dots + (-1)^{n-1} n x  + (-1)^n \\
&= \sum_{k=0}^n (-1)^k \binom{n}{k} x^{n-k}. 
\end{align*}

\begin{exercise}\label{ex:binomial1-1}
Write $x=1$, $y=-1$ into the Binomial theorem. 
What do you observe? 
\end{exercise}

\begin{exercise}\label{ex:expandusingbinomialthm}
Expand the following expressions: 
$(x+y)^8$, $(x-y)^8$, $(a+1)^{10}$, $(b-3)^5$, $(1 + 2/x)^5$, $\left( a + b \right)^6$, $\left( 1 + x \right)^5$, $\left(3a + 4b \right)^4$, $\left( 3 - 2x \right)^4$. 
\end{exercise}

\begin{exercise}\label{ex:coefficientinbinomialthm}
In the binomial expansion of $\left( 1- x/2 \right)^9$, written in terms of \emph{ascending} powers of $x$, 
find the fourth term.
Then find the coefficient of $x^5$. 
\end{exercise}

%\begin{proof}[Second proof of Theorem~\ref{thm:binomial}]
%Consider $(x+y)^n$ as the $n$-fold product of $(x+y)$ by itself: 
%\[
%(x+y)^n = (x+y) \cdot (x+y) \cdot \dots \cdot (x+y). 
%\]
%\end{proof}



\section{Identities}



In this Section we investigate several properties of Pascal's triangle. 
Throughout this Section, 
we will first conjecture what identities hold by looking at the first 12 rows of Pascal's triangle. 
Therefore solving Exercise~\ref{ex:pascal12} is essential before continuing. 

Let us start by the sum of the numbers in a row: 
\begin{align*}
1 & = 1, \\
1 + 1 &= 2, \\
1 + 2 + 1 &= 4, \\
1 + 3 + 3 + 1 &= 8, \\
1 + 4 + 6 + 4 + 1 &= 16, \\
1 + 5 + 10 + 10 + 5 + 1 &= 32, \\
1 + 6 + 15 + 20 + 15 + 6 + 1 &= 64. 
\end{align*}

It seems from these equations that the sum of the numbers in the $n$th row is $2^n$. 
This stetement is equivalent to the equality
\begin{equation} %\label{eq:sumofbinomial}
\notag \binom{n}{0} + \binom{n}{1} + \binom{n}{2} + \dots + \binom{n}{n-2} + \binom{n}{n-1} + \binom{n}{n} = 2^n. 
\end{equation}
Note, that we have already proved this, first in Proposition~\ref{prop:sumofbinomial}, then later in Exercise~\ref{ex:binomial1+1}.
Now, we prove it a third way, using the generating rule of Pascal's triangle. 

Let us consider first the 7th row, 
and try to compute the sum using the generating rule of Pascal's triangle, 
rather than adding the numbers: 
\begin{align*}
& 1 + 7 + 21 + 35 + 35 + 21 + 7 + 1 \\
&= 1 + (1 + 6) + (6 + 15) + (15 + 20) + (20 + 15) + (15 + 6) + (6 + 1) + 1 \\
&= 2 \cdot (1 + 6 + 15 + 20 + 15 + 6 + 1 ) = 2 \cdot 2^6 = 2^7 = 128. 
\end{align*}
This idea can be used in the general case, as well. 
%One only needs to use induction on $n$. 

Now, we prove that the sum of the numbers in the $n$th row of Pascal's triangle is $2^n$ by induction on $n$. 
The statement holds for $n=0$ and $n=1$ (in fact, we just calculated that it holds for $n\leq 7$). 
Assume now that the statement holds for $n$, as well. 
That is, the sum of the numbers in the $n$th row is $2^{n}$. 
Consider the sum of the $(n+1)$st row, 
and let us use the generating rule of Pascal's triangle: 
\begin{align}
\notag & \binom{n+1}{0} + \binom{n+1}{1} + \binom{n+1}{2} + \dots + \binom{n+1}{n-1} + \binom{n+1}{n} + \binom{n+1}{n+1} \\
\notag &= \binom{n}{0} + \left( \binom{n}{0} + \binom{n}{1} \right) + \left( \binom{n}{1} + \binom{n}{2} \right)  + \left( \binom{n}{2} + \binom{n}{3} \right) 
+ \dots \\
\notag &+ %\left( \binom{n}{n-3} + \binom{n}{n-3} \right)  + 
\left( \binom{n}{n-2} + \binom{n}{n-1} \right)  + \left( \binom{n}{n-1} + \binom{n}{n} \right) + \binom{n}{n} \\
%\notag &= \binom{n}{0} + 2 \cdot \left[ \binom{n}{1} + \binom{n}{2} + \binom{n}{3} + \dots + \binom{n}{n-2} + \binom{n}{n-1} \right] + \binom{n}{n} \\
\notag &= 2 \cdot \left[ \binom{n}{0} + \binom{n}{1} + \binom{n}{2} + \dots + \binom{n}{n-2} + \binom{n}{n-1} + \binom{n}{n} \right] \\
\notag &= 2 \cdot 2^{n} = 2^{n+1}.
\end{align}
First, we replaced $\binom{n+1}{0}=1$ by $\binom{n}{0}=1$, 
and $\binom{n+1}{n+1} = 1$ by $\binom{n}{n} = 1$, then we used the generating rule of Pascal's triangle. 
Then we observed that every $\binom{n}{k}$ occurs twice in the sum (for $0\leq k\leq n$). 
Finally, we used the induction hypothesis on the sum of the numbers for the $n$th row.  

Let us use a similar reasoning to calculate the sum of the numbers in a row, \emph{with alternating signs}. 
That is, compute the sum
\[
\sum_{k=0}^n \left( -1 \right)^k \cdot \binom{n}{k} = \binom{n}{0} - \binom{n}{1} + \binom{n}{2} - \dots + \left( -1 \right)^{n-1} \cdot \binom{n}{n-1} + \left( -1 \right)^{n} \cdot \binom{n}{n}. 
\]
It is easy to compute this sum for the first couple rows: 
\begin{align*}
1 & = 1, \\
1 - 1 &= 0, \\
1 - 2 + 1 &= 0, \\
1 - 3 + 3 - 1 &= 0, \\
1 - 4 + 6 - 4 + 1 &= 0, \\
1 - 5 + 10 - 10 + 5 - 1 &= 0, \\
1 - 6 + 15 - 20 + 15 - 6 + 1 &= 0, \\
1 - 7 + 21 - 35 + 35 - 21 + 7 - 1 &= 0.
\end{align*}
It seems likely that for $n\geq 1$ the alternating sum of the numbers in the $n$th row of Pascal's triangle is 0. 
\begin{exercise}\label{ex:alternatingsum}
The alternating sum of the $n$th row is clearly 0 if $n$ is odd. 
Why? 
\end{exercise}
Let us try to use the former argument to compute the alternating sum of the numbers in the 8th row: 
\begin{align*}
& 1 - 8 + 28 - 56 + 70 - 56 + 28 - 8 + 1 = 1 - (1 + 7) + (7 + 21) - (21 + 35) \\
&+ (35 + 35) - (35 + 21) + (21+7) - (7 + 1) + 1 = (1-1) + (-7+7) \\
&+ (21-21) + (-35 + 35) + (35-35) + (-21 + 21) + (7-7) + (-1+1) = 0. 
\end{align*}
Using the very same proof technique, we can prove that the alternating sum is 0 in the $n$th row, as well. 
\begin{exercise}\label{ex:alternatingsum2}
Prove that the alternating sum of the $n$th row is 0, 
that is,  
\[
\sum_{k=0}^n \left( -1 \right)^k \cdot \binom{n}{k} = \binom{n}{0} - \binom{n}{1} + \dots + \left( -1 \right)^{n-1} \cdot \binom{n}{n-1} + \left( -1 \right)^{n} \cdot \binom{n}{n} = 0. 
\]
\end{exercise}
In fact, the technique can be used to prove an even more general statement, 
namely we can determine the alternating sum if we stop at the $k$th term (for some $k\leq n-1$): 

\begin{proposition}\label{prop:alternatingsum}
\[
\binom{n}{0} - \binom{n}{1} + \dots + \left( -1 \right)^{k-1} \cdot \binom{n}{k-1} + \left( -1 \right)^{k} \cdot \binom{n}{k} = \left( - 1 \right)^k \cdot \binom{n-1}{k}. 
\]
\end{proposition}

\begin{proof}
Consider the alternating sum of the $n$th row (for $n\geq 1$), 
and use the generating rule of Pascal's triangle: 
\begin{align*}
& \binom{n}{0} - \binom{n}{1} + \binom{n}{2} - \dots + \left( -1 \right)^{k-1} \cdot \binom{n}{k-1} + \left( -1 \right)^{k} \cdot \binom{n}{k}  \\
&= \binom{n-1}{0} - \left( \binom{n-1}{0} + \binom{n-1}{1} \right) + \left( \binom{n-1}{1} + \binom{n-1}{2} \right)  - \dots \\
%\left( \binom{n}{2} + \binom{n}{3} \right) 
%+ \dots \\
&+ %(-1)^{n-1} \cdot \left( \binom{n}{n-2} + \binom{n}{n-1} \right)  + 
(-1)^{k-1} \cdot \left( \binom{n-1}{k-2} + \binom{n-1}{k-1} \right) + (-1)^{k} \cdot \left( \binom{n-1}{k-1} + \binom{n-1}{k} \right) \\
%\notag &= \binom{n}{0} + 2 \cdot \left[ \binom{n}{1} + \binom{n}{2} + \binom{n}{3} + \dots + \binom{n}{n-2} + \binom{n}{n-1} \right] + \binom{n}{n} \\
&= \left( \binom{n-1}{0} - \binom{n-1}{0} \right) + \left( - \binom{n-1}{1} + \binom{n-1}{1} \right) + \left( \binom{n-1}{2} - \binom{n-1}{2} \right) + \dots \\
%&+ \left( \left( -1 \right)^{n-2} \cdot \binom{n}{n-2} + \left( -1 \right)^{n-1} \cdot \binom{n}{n-2} \right)  \\
&+ \left( \left( -1 \right)^{k-2} \cdot \binom{n-1}{k-2} + \left( -1 \right)^{k-1} \cdot \binom{n-1}{k-2} \right)  \\
&+ \left( \left( -1 \right)^{k-1} \cdot \binom{n-1}{k-1} + \left( -1 \right)^{k} \cdot \binom{n-1}{k-1} \right) + \left( -1 \right)^k \cdot \binom{n-1}{k} \\
&= 0 + 0 + 0 + \dots + 0 + 0 + \left( -1 \right)^k \cdot \binom{n-1}{k} = \left( -1 \right)^k \cdot \binom{n-1}{k}. 
\end{align*}
First, we replaced $\binom{n}{0}=1$ by $\binom{n-1}{0}=1$, 
and $\binom{n}{n} = 1$ by $\binom{n-1}{n-1} = 1$, then we used the generating rule of Pascal's triangle. 
Then we observed that every $\binom{n-1}{j}$ occurs twice in the sum: first with a positive sign, then right after it with a negative sign 
(for $0\leq j\leq k-1$). 
The only term remaining is $\left( -1 \right)^k \cdot \binom{n-1}{k}$. 
\end{proof}

If we define $\binom{n-1}{n}$ to be 0 
(considering there are no $n$-element subsets of an $(n-1)$-element set), 
then our statement on the alternating sums follows from Proposition~\ref{prop:alternatingsum}. 

Now, consider the sum of the squares of the numbers in a row. 
We can find a pattern here, as well: 
\begin{align*}
1^2 & = 1, \\
1^2 + 1^2 &= 2, \\
1^2 + 2^2 + 1^2 &= 6, \\
1^2 + 3^2 + 3^2 + 1^2 &= 20, \\
1^2 + 4^2 + 6^2 + 4^2 + 1^2 &= 70, \\
1^2 + 5^2 + 10^2 + 10^2 + 5^2 + 1^2 &= 252, \\
1^2 + 6^2 + 15^2 + 20^2 + 15^2 + 6^2 + 1^2 &= 924. 
\end{align*}
After computing the first twelve rows of Pascal's triangle in Exercise~\ref{ex:pascal12}, 
we can observe that the results are the numbers occurring in the middle column. 
That is, we can conjecture that the sum of the square of the numbers in row $n$ is $\binom{2n}{n}$, 
that is, 

\begin{proposition}\label{prop:sumsquaresofrow}
\begin{equation}\label{eq:sumsquaresofrow}
\sum_{k=0}^{n} \binom{n}{k}^2 = \binom{n}{0}^2 + \binom{n}{1}^2 + \dots + \binom{n}{n-1}^2 + \binom{n}{n}^2 = \binom{2n}{n}. 
\end{equation}
\end{proposition}

\begin{proof}
As earlier, we try to understand why this equation holds by giving a combinatorial meaning to both sides. 
The right hand side gives away a clue: $\binom{2n}{n}$ is the number of ways to choose $n$ elements out of a $2n$-element set 
(say $S = \halmaz{1, 2, \dots , 2n}$). 
Our plan is to prove that the left hand side of \eqref{eq:sumsquaresofrow} is the number of $n$-element subsets of $S$, as well. 
Let $S_1 = \halmaz{1, 2, \dots , n}$ and $S_2 = \halmaz{n+1, n+2, \dots , 2n}$. 
Now, try to count the number of ways to choose $n$-element of $S$ by counting how many elements we choose from $S_1$ and from $S_2$. 
If we choose 0 element from $S_1$, then we must choose $n$ elements from $S_2$. 
We can do this in $\binom{n}{0} \cdot \binom{n}{n}$-many ways. 
If we choose 1 element from $S_1$, then we must choose $n-1$ elements from $S_2$. 
We can do this in $\binom{n}{1} \cdot \binom{n}{n-1}$-many ways. 
If we choose 2 elements from $S_1$, then we must choose $n-2$ elements from $S_2$. 
We can do this in $\binom{n}{2} \cdot \binom{n}{n-2}$-many ways. 
In general, if we choose k elements from $S_1$, then we must choose $n-k$ elements from $S_2$. 
We can do this in $\binom{n}{k} \cdot \binom{n}{n-k}$-many ways. 
In the end, if we choose $n$ elements from $S_1$, then we must choose $0$ element from $S_2$. 
We can do this in $\binom{n}{n} \cdot \binom{n}{0}$-many ways. 
Thus, choosing $n$ elements out of $2n$ can be done in the following number of ways: 
\[
\binom{n}{0} \cdot \binom{n}{n} + \binom{n}{1} \cdot \binom{n}{n-1} + \dots %+ \binom{n}{n-1} \cdot \binom{n}{1} 
+ \binom{n}{n}\cdot \binom{n}{0} = \sum_{k=0}^n \binom{n}{k} \cdot \binom{n}{n-k}. 
\]
Finally, 
let us rewrite the left hand side by using the symmetry of Pascal's triangle, 
that is, $\binom{n}{n-k} = \binom{n}{k}$ to obtain the left hand side of \eqref{eq:sumsquaresofrow}: 
\begin{align*}
& \binom{n}{0} \cdot \binom{n}{n} + \binom{n}{1} \cdot \binom{n}{n-1} + \dots %+ \binom{n}{n-1} \cdot \binom{n}{1} 
+ \binom{n}{n}\cdot \binom{n}{0} = \sum_{k=0}^n \binom{n}{k} \cdot \binom{n}{n-k} \\
&= \sum_{k=0}^{n} \binom{n}{k}^2 = \binom{n}{0}^2 + \binom{n}{1}^2 + \dots + \binom{n}{n-1}^2 + \binom{n}{n}^2. 
\end{align*}
That is, both sides of \eqref{eq:sumsquaresofrow} counts the number of ways of choosing $n$ elements out of a $2n$-element set 
(or alternatively, the number of $n$-element subsets of a $2n$-element set), 
and therefore must be equal. 
\end{proof}

This idea can be used in a more general setting. 
\begin{exercise}\label{ex:n+mchoosek}
Prove that 
\begin{align}\label{eq:n+mchoosek}
\notag & \sum_{k=0}^l \binom{n}{k} \cdot \binom{m}{l-k} = \binom{n+m}{l}, \text{ that is, } \\
& \binom{n}{0} \cdot \binom{m}{l} + \binom{n}{1} \cdot \binom{m}{l-1} + \dots + \binom{n}{l} \cdot \binom{m}{0}= \binom{n+m}{l}. 
\end{align}
How do we need to choose $m$ and $l$ so that \eqref{eq:n+mchoosek} gives us the equality \eqref{eq:sumsquaresofrow}? 
\end{exercise}

We could have used the Binomial theorem to prove \eqref{eq:sumsquaresofrow}: 
\begin{proof}[Second proof of Proposition~\ref{prop:sumsquaresofrow}]
Consider $(x+y)^{2n}$, and expand it using the Binomial theorem: 
\[
(x+y)^{2n} = \sum_{k=0}^{2n} \binom{2n}{k} x^{2n-k} \cdot y^{k}. 
\]
Then the right hand side of \eqref{eq:sumsquaresofrow} is the coefficient of the term $x^ny^n$. 
We prove that the left hand side is the coefficient of $x^ny^n$, as well. 
For this, we compute $(x+y)^{2n}$ by multiplying $(x+y)^n \cdot (x+y)^n$ after expanding both factors using the Binomial theorem: 
\[
(x+y)^{2n} = (x+y)^n \cdot (x+y)^n = \left( \sum_{k=0}^n \binom{n}{k} x^{n-k}y^k \right) \cdot \left( \sum_{k=0}^n \binom{n}{k} x^{n-k}y^k \right). 
\]
Now, let us compute the coefficient of $x^n y^n$. 
When do we obtain $x^n y^n$ when we multiply $\left( \sum_{k=0}^n \binom{n}{k} x^{n-k}y^k \right) $ by itself? 
Take for example $x^n$ from the first factor, this must be multiplied by $y^n$ from the second factor to obtain $x^n y^n$. 
The coefficient of $x^n$ in the first factor is $\binom{n}{0}$, the coefficient of $y^n$ in the second factor is $\binom{n}{n}$, 
thus this multiplication contributes by $\binom{n}{0} \cdot \binom{n}{n}$ to the coefficient of $x^n y^n $ in $(x+y)^{2n}$. 
Similarly, take the term $x^{n-1}y$ from the first factor, this must be multiplied by $xy^{n-1}$ from the second factor to obtain $x^n y^n$. 
The coefficient of $x^{n-1}y$ in the first factor is $\binom{n}{1}$, the coefficient of $xy^{n-1}$ in the second factor is $\binom{n}{n-1}$, 
thus this multiplication contributes by $\binom{n}{1} \cdot \binom{n}{n-1}$ to the coefficient of $x^n y^n $ in $(x+y)^{2n}$. 
In general, for some $k$ the term $x^{n-k}y^k$ in the first factor must be multiplied by $x^k y^{n-k}$ from the second factor. 
The coefficient of $x^{n-k}y^k$ in the first factor is $\binom{n}{k}$, the coefficient of $x^ky^{n-k}$ in the second factor is $\binom{n}{n-k}$, 
thus this multiplication contributes by $\binom{n}{k} \cdot \binom{n}{n-k}$ to the coefficient of $x^n y^n $ in $(x+y)^{2n}$. 
That is, the coefficient of $x^n y^n$ in $(x+y)^{2n}$ is 
\[
\sum_{k=0}^n \binom{n}{k} \cdot \binom{n}{n-k}. 
\]
Moreover, the coefficient of $x^ny^n$ in $(x+y)^{2n}$ is $\binom{2n}{n}$, thus the two numbers must be equal. 
Applying the symmetry of Pascal's triangle (that is, $\binom{n}{k} = \binom{n}{n-k}$), 
we obtain \eqref{eq:sumsquaresofrow}:
\[
\sum_{k=0}^n \binom{n}{k}^2 = \sum_{k=0}^n \binom{n}{k} \cdot \binom{n}{n-k} = \binom{2n}{n}. 
\]
\end{proof}

\begin{exercise}\label{ex:n+mchoosek2}
Solve Exercise~\ref{ex:n+mchoosek} using the Binomial theorem. 
\end{exercise}

After dealing with sums of rows, 
consider sums where we move diagonally upwards. 
That is, when we sum up the $m$th elements of every row. 
For $m=0$ it is pretty easy: 
\[
\binom{n}{0} + \binom{n-1}{0} + \dots + \binom{1}{0} + \binom{0}{0} = n+1. 
\]
For $m=1$ we have 
\[
\binom{n}{1} + \binom{n-1}{1} + \dots + \binom{2}{1} + \binom{1}{1} = n + (n-1) + \dots + 2 + 1 = \frac{n\cdot (n+1)}{2}, 
\]
by Proposition~\ref{prop:sumk}. 

For $m=2$ it is a bit harder to do the calculations, but still manageable: 
\begin{align*}
&\binom{n}{2} + \binom{n-1}{2} + \dots + \binom{3}{2} + \binom{2}{2} \\
&= \frac{n \cdot (n-1)}{2} + \frac{(n-1) \cdot (n-2)}{2} + \dots + \frac{3 \cdot 2}{2} + \frac{2 \cdot 1}{2} \\
&= \frac12 \cdot \left( n \cdot (n-1) + (n-1) \cdot (n-2) + \dots + 3 \cdot 2 + 2 \cdot 1 \right) \\
&= \frac12 \cdot \frac{(n+1) \cdot n \cdot (n-1)}{3} = \frac{(n+1) \cdot n \cdot (n-1)}{3 \cdot 2 \cdot 1}. 
\end{align*}
Here, we used Exercise~\ref{ex:sumk(k+1)} to calculate the sum $\sum_{i=1}^{n-1} i \cdot (i+1)$. 

It is quite clear that by increasing $m$, we would have harder and harder time to calculate the obtained sums. 
Nevertheless, only by computing the first couple sums we can make a guess at the general answer: 
\begin{align*}
& \text{for $m=0$} & \sum_{k=0}^n \binom{k}{0} &= n+1, \\
& \text{for $m=1$} & \sum_{k=1}^n \binom{k}{1} &= \frac{(n+1) \cdot n}{2}, \\
& \text{for $m=2$} & \sum_{k=2}^n \binom{k}{2} &= \frac{(n+1) \cdot n \cdot (n-1)}{3 \cdot 2}. 
\end{align*}
Now, hold on for a second! 
The right hand sides here are $\binom{n+1}{1}$, $\binom{n+1}{2}$, $\binom{n+1}{3}$, respectively. 
From this, we may conjecture that in general the sum $\sum_{k=m}^n \binom{k}{m}$ will be $\binom{n+1}{m+1}$. 
This is indeed the case. 
\begin{proposition}\label{prop:sumkchoosem}
The sum of $m$th elements of Pascal's triangle is $\binom{n+1}{m+1}$, that is, 
\begin{equation}\label{eq:sumkchoosem}
\sum_{k=m}^n \binom{k}{m} = \binom{m}{m} + \binom{m+1}{m} + \dots + \binom{n}{m} = \binom{n+1}{m+1}. 
\end{equation}
\end{proposition}

\begin{proof}
We prove the proposition by induction on $n$. 
Fix $m$ first, 
then the induction starts by checking if the statement holds for the smallest possible $n$, that is, for $n = m$. 
For $n=m$ the left hand side is simply $\binom{m}{m} = 1$, 
the right hand side is $\binom{m+1}{m+1} = 1$, 
and the statement holds. 
Let us assume now that the statement holds for $n-1$, 
that is, 
\[
\sum_{k=m}^{n-1} \binom{k}{m} = \binom{m}{m} + \binom{m+1}{m} + \dots + \binom{n-1}{m} = \binom{n}{m+1}. 
\]
This is the induction hypothesis. 
Now we prove that the statement holds for $n$, as well. 
Consider the sum $\sum_{k=m}^{n} \binom{k}{m}$: 
\begin{align*}
& \sum_{k=m}^{n} \binom{k}{m} = \underbrace{\binom{m}{m} + \binom{m+1}{m} + \dots + \binom{n-1}{m}}_{= \binom{n}{m+1}, \text{ by the induction hypothesis}} + \binom{n}{m} \\
&= \binom{n}{m+1} + \binom{n}{m} = \binom{n+1}{m+1}. 
\end{align*}
Here, we first used the induction hypothesis, 
then the generating rule of Pascal's triangle (Proposition~\ref{prop:binomsum}). 
\end{proof}

Again, 
the induction proof clearly settles that our conjecture was true, 
but it does not clarify the reason why this identity holds. 
By finding combinatorial meaning to both sides of \eqref{eq:sumkchoosem}, 
we can understand what is ``behind the curtain''. 

\begin{proof}[Second proof of Proposition~\ref{prop:sumkchoosem}]
Again, the right hand side gives a clue on what we need to find. 
Since $\binom{n+1}{m+1}$ is the number of ways choosing $m+1$ elements out of an $n$-element set, 
this is what we will try to find on the left hand side, as well. 
Let $S= \halmaz{1, 2, \dots , n, n+1}$. 
Try to choose $m+1$ elements in the following way: first choose the \emph{largest} one, 
then choose the remaining $m$ elements. 
Clearly the largest is at least $m+1$. 
If we choose $m+1$ as the largest chosen number, 
then we need to choose $m$ elements out of the $m$-element set $\halmaz{1, 2, \dots , m}$, 
this can be done in $\binom{m}{m}$-many ways. 
If we choose $m+2$ as the largest chosen number, 
then we need to choose $m$ elements out of the $(m+1)$-element set $\halmaz{1, 2, \dots , m+1}$, 
this can be done in $\binom{m+1}{m}$-many ways. 
If we choose $m+3$ as the largest chosen number, 
then we need to choose $m$ elements out of the $(m+2)$-element set $\halmaz{1, 2, \dots , m+2}$, 
this can be done in $\binom{m+2}{m}$-many ways. 
In general, if we choose $k+1$ as the largest chosen number (for some $m \leq k\leq n$), 
then we need to choose $m$ elements out of the $k$-element set $\halmaz{1, 2, \dots , k}$, 
this can be done in $\binom{k}{m}$-many ways. 
If we choose $n$ as the largest chosen number, 
then we need to choose $m$ elements out of the $(n-1)$-element set $\halmaz{1, 2, \dots , n-1}$, 
this can be done in $\binom{n-1}{m}$-many ways. 
Finally, if we choose $n+1$ as the largest chosen number, 
then we need to choose $m$ elements out of the $n$-element set $\halmaz{1, 2, \dots , n}$, 
this can be done in $\binom{n}{m}$-many ways. 
That is, the number of ways we can choose $(m+1)$ elements out of an $(n+1)$-element set is 
\[
\sum_{k=m}^n \binom{k}{m} = \binom{m}{m} + \binom{m+1}{m} + \dots + \binom{n}{m}
\]
on the one hand, and $\binom{n+1}{m+1}$ on the other hand. 
Thus the two numbers must be equal, as they count the same thing. 
Hence, 
\[
\sum_{k=m}^n \binom{k}{m} = \binom{m}{m} + \binom{m+1}{m} + \dots + \binom{n}{m} = \binom{n+1}{m+1}. 
\]
\end{proof}

Note, that from this identity we immediately obtain a formula for the sum of integer numbers and for the sum of squares. 
Indeed, 
\[
1 + 2 + \dots + n = \sum_{k=1}^n {k} = \sum_{k=1}^n \binom{k}{1} = \binom{n+1}{2} = \frac{(n+1) \cdot n}{2}, 
\]
because $k$ in $\sum_{k=1}^n k$ can be expressed as $\binom{k}{1}$. 
Similarly, $k^2 = \binom{k+1}{2} + \binom{k}{2}$ by Exercise~\ref{ex:n^2binom} (for $k\geq 2$). 
Thus 
\begin{align*}
1^2 + 2^2 + \dots + n^2 &= \sum_{k=1}^n k^2 = 1 + \sum_{k=2}^n \left(\binom{k+1}{2} + \binom{k}{2}\right) \\
&= 1 + \sum_{k=2}^n \binom{k+1}{2} + \sum_{k=2}^n \binom{k}{2} \\
&= \binom{2}{2} + \sum_{k=2}^n \binom{k+1}{2} + \sum_{k=2}^n \binom{k}{2} \\
&= \sum_{k=1}^{n} \binom{k+1}{2} + \sum_{k=2}^n \binom{k}{2} = \sum_{k=2}^{n+1} \binom{k}{2} + \sum_{k=2}^n \binom{k}{2} \\
&= \binom{n+2}{3} + \binom{n+1}{3} \\
&= \frac{(n+2) \cdot (n+1) \cdot n}{3 \cdot 2} + \frac{(n+1) \cdot n \cdot (n-1)}{3 \cdot 2} \\
&= \frac{n \cdot (n+1)}{3 \cdot 2} \cdot \left( (n+2)+ (n-1) \right) \\
&= \frac{n \cdot (n+1) \cdot (2n+1)}{6}. 
\end{align*}

\begin{exercise}\label{ex:diagonal}
Prove a similar identity for summing up numbers diagonally in the other direction:
\begin{equation}%\label{eq:diagonal}
\notag \sum_{k=0}^m \binom{n+k}{k} = \binom{n}{0} + \binom{n+1}{1} + \dots + \binom{n+m}{m} = \binom{n+m+1}{m}. 
\end{equation}
\end{exercise}

\begin{exercise}\label{ex:rowp}
Let $p$ be a prime. 
Prove that every number in row $p$ (except for the first and last) is divisible by $p$. 
By observing the first 12 rows of Pascal's triangle, 
confirm that this property does not necessarily hold if $p$ is not a prime. 
\end{exercise}
