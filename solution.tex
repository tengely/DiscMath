% !TEX root = lectnote.tex
% !TEX spellcheck = en_GB-oed 

\chapter{Solutions}\label{cha:solutions}

\section{Introduction}
\begin{enumerate}
\item[\ref{intro-ex-1}]
There are three given sets $A=\halmaz{3,4,6,7,8},B=\halmaz{2,4,5,6,8}$ and $C=\halmaz{1,2,4,5,8}$. 
We have that
\begin{align*}
A\setminus B&=\halmaz{3,7}\\
C\cap B&=\halmaz{2,4,5,8}.
\end{align*}
Thus 
$$
(A\setminus B)\cup(C\cap B)=\halmaz{2,3,4,5,7,8}.
$$

\item[\ref{intro-ex-2}]
We have three sets $A=\halmaz{1,3,4,6,7},B=\halmaz{2,4,5,6,8}$ and $C=\halmaz{1,3,4,5,8}$. 
\begin{align*}
(A\cap B)&=\halmaz{4,6}\\
(C\cap B)&=\halmaz{4,5,8}.
\end{align*}
Therefore
$$
(A\cap B)\setminus(C\cap B)=\halmaz{6}.
$$

\item[\ref{intro-ex-3}]
Now the three given sets are $A=\halmaz{1,3,4,6,7},B=\halmaz{2,4,6,8}$ and $C=\halmaz{1,3,4,8}$. 
\begin{align*}
(A\setminus B)&=\halmaz{1,3,7}\\
(C\setminus B)&=\halmaz{1,3}.
\end{align*}
So we obtain
$$
(A\setminus B)\cup(C\setminus B)=\halmaz{1,3,7}.
$$

\item[\ref{intro-ex-4}]

(a) The elements of the set are 7, 10 and 13.

(b) The elements of the set are 0, 1 and 4.

(c) The possible differences are $3-1, 3-2, 4-1, 4-2, 5-1$ and $5-2$, thus the elements of the set are 1, 2, 3 and 4.

\item[\ref{intro-ex-5}]

(a) $\halmazvonal{2k}{ k\in\halmaz{1,2,3,4,5}}$, 

(b) $\halmazvonal{k^2}{k\in\halmaz{1,2,3,4,5}}$, 

(c) $\halmazvonal{2^{-k}}{ k\in\mathbb{N}\cup\halmaz{0}}$,

(d) $\halmazvonal{a/b }{a,b\in\mathbb{N}, b\leq a\leq 2b}$. 

\item[\ref{intro-ex-6}]

(a) 
\begin{center}
\begin{venndiagram3sets}
\fillACapB\fillC
\end{venndiagram3sets}
\end{center}

(b)
\begin{center}
\begin{venndiagram3sets}
\fillANotB\fillANotC
\end{venndiagram3sets}
\end{center}

(c)
\begin{center}
\begin{venndiagram3sets}
\fillACapC\fillBCapC
\end{venndiagram3sets}
\end{center}

(d)
\begin{center}
\begin{venndiagram3sets}
\fillACapC\fillBCapC\fillACapB
\end{venndiagram3sets}
\end{center}

(e)
\begin{center}
\begin{venndiagram3sets}
\fillACapBNotC\fillBCapCNotA\fillACapCNotB
\end{venndiagram3sets}
\end{center}

(f)
\begin{center}
\begin{venndiagram3sets}
\fillANotB\fillBNotC\fillCNotA
\end{venndiagram3sets}
\end{center}

\item[\ref{intro-ex-7}]
The set $A\cap B\cap C$ is a subset of all other sets for which we have certain cardinality conditions, so we may set
$$
A\cap B\cap C=\halmaz{1,2}.
$$
The conditions for $|A\cap B|, |A\cap C|$ and $|B\cap C|$ are satisfied. We have that $|A|=4$, that means that two elements
are missing from $A\setminus(B\cup C)$. We let $A\setminus(B\cup C)=\halmaz{3,4}$. Similarly for $B\setminus(A\cup C)$ and $C\setminus(A\cup B)$. We obtain that
\begin{center}
\begin{venndiagram3sets}[labelABC={\tiny{1,2}},labelOnlyA={\tiny{3,4}},labelOnlyB={\tiny{5,6}},labelOnlyC={\tiny{7,8}}]
\end{venndiagram3sets}
\end{center}

\item[\ref{intro-ex-8}]
Following the solution of Exercise \ref{intro-ex-7} we get:
\begin{center}
\begin{venndiagram3sets}[labelABC={\tiny{1,2}},labelOnlyBC={\tiny{3}},labelOnlyA={\tiny{4,5}},labelOnlyB={\tiny{6,7}},labelOnlyC={\tiny{8,9,10}}]
\end{venndiagram3sets}
\end{center}

\item[\ref{intro-ex-9}]

(a) $\sum_{i=4}^7 i=4+5+6+7$,

(b) $\sum_{i=1}^5 (i^2-i)=0+2+6+12+20$,

(c) $\sum_{i=1}^4 10^i=10+100+1000+10000$,

(d) $\sum_{2\leq i\leq 5} \frac{1}{2^i}=\frac{1}{4}+\frac{1}{8}+\frac{1}{16}+\frac{1}{32}$,

(e) $\sum_{i\in S} (-1)^i$, where $S=\halmaz{2,3,5,8}$ is $1+(-1)+(-1)+1$.



\item[\ref{intro-ex-10}]

(a) $2+4+6+8+10=\sum_{i=1}^5 2i$,

(b) $1+4+7+10=\sum_{i=0}^3 (3i+1)$,

(c) $\frac{1}{4}+\frac{1}{2}+1+2+4=\sum_{i=-2}^2 2^i$,

(d) $\frac{1}{4}-\frac{1}{2}+1-2+4=\sum_{i=-2}^2 (-2)^i$.

\item[\ref{intro-ex-11}]

(a) $\prod_{i=-4}^{-1} i=(-4)\cdot(-3)\cdot(-2)\cdot(-1)$,

(b) $\prod_{i=1}^4 (i^2)=1\cdot 4\cdot 9\cdot 16$,

(c) $\prod_{i=1}^3 2^i=2\cdot 4\cdot 8$,

(d) $\prod_{-2\leq i\leq 3} \frac{1}{2^i}=4\cdot 2\cdot 1\cdot \frac{1}{2}\cdot \frac{1}{4}\cdot \frac{1}{8}$,

(e) $\prod_{i\in S} (-1)^i$, where $S=\halmaz{2,4,6,7}$ is $(-1)^2\cdot(-1)^4\cdot(-1)^6\cdot(-1)^7$.

\item[\ref{intro-ex-12}]

(a) $1\cdot 3\cdot 5\cdot 7=\prod_{i=0}^3 (2i+1)$,

(b) $(-1)\cdot 2\cdot 5\cdot 8=\prod_{i=0}^3 (3i-1)$,

(c) $\frac{1}{9}\cdot\frac{1}{3}\cdot 1\cdot 3\cdot 9=\prod_{i=-2}^2 3^{i}$.




\item[\ref{ex:factorial1}]
The values are
\begin{align*}
0! &= 1, \\
1! &= 1, \\
2! &= 2, \\
3! &= 6, \\
4! &= 24, \\
5! &= 120, \\
6! &= 720, \\
7! &= 5~040, \\
8! &= 40~320.
\end{align*}

\item[\ref{ex:factorial2}]
The values are
\begin{align*}
5+3! &= 5+6 = 11, \\
(5+3)! = 8! &= 40~320, \\
4-2\cdot 3! &= 4-2\cdot 6 = 4-12=-8, \\
(4-2)\cdot 3! &= (4-2) \cdot 6 = 2 \cdot 6 = 12, \\
4 - (2 \cdot 3)! &= 4 - 6! = 4 - 720 = -716, \\
3 \cdot 2! &= 3 \cdot 2 = 6, \\
(3 \cdot 2)! &= 6! = 720, \\
4 \cdot 3! &= 4 \cdot 6 = 24, \\
4! \cdot 5 &= 24 \cdot 5 = 120. 
\end{align*}

\item[\ref{ex:factorial3}]
Let 
\[
S_n = \halmazvonal{k}{k \text{ is a positive integer}, k\leq n} = \halmaz{1, 2, \dots , n}. 
\]
Then it is easy to see that $S_n = S_{n-1} \cup \halmaz{n}$, 
that is, $S_n$ is the disjoint union of $S_{n-1}$ and $\halmaz{n}$. 
Then by the definition of the factorial, 
we have 
\[
n! = \prod_{k \in S_n} k = \left( \prod_{k \in \halmaz{n}} k \right) \cdot \left( \prod_{k \in S_{n-1}} k \right) = n \cdot (n-1)!. 
\]
If $n \geq 2$, then another proof could be 
\[
n! = n \cdot \underbrace{(n-1) \cdot (n-2) \cdot \dots \cdot 2 \cdot 1 }_{(n-1)!}= n \cdot (n-1)!. 
\]
Nevertheless, the claim is true for $n=1$, as well: 
\[
1! = 1 = 1 \cdot 1 = 1 \cdot 0!. 
\]


\item[\ref{intro-ex-13}]

(a) We obtain that
\begin{align*}
678 &= 1\cdot 567+111\\
567 &= 5\cdot 111+12\\
111 &= 9\cdot 12+3\\
12 &= 4\cdot 3+0.
\end{align*}
Thus $\gcd(678,567)=3$. We work backwards to compute $x$ and $y:$
\begin{align*}
3&=111-9\cdot 12\\
 &=111-9\cdot (567-5\cdot 111)=-9\cdot 567+46\cdot 111\\
 &=-9\cdot 567+46 \cdot (678-567)=46\cdot 678-55\cdot 567.
\end{align*}
Hence we have
$$
46\cdot 678-55\cdot 567=\gcd(678,567)=3.
$$

(b) We get that
\begin{align*}
803 &= 2\cdot 319+165\\
 319 &= 1\cdot 165+154\\
 165 &= 1\cdot 154+11\\
 154 &= 14\cdot 11+0.
\end{align*}
It follows that $\gcd(803,319)=11$. Now we find $x$ and $y:$
\begin{align*}
11&= 165-154\\
  &= 165-(319-165)=-319+2\cdot 165\\
  &= -319+2 \cdot (803-2\cdot 319)=2\cdot 803-5\cdot 319.
\end{align*}
So we get the equation
$$
2\cdot 803-5\cdot 319=\gcd(803,319)=11.
$$

(c) In this case the computations go as follows
\begin{align*}
2701 &= 1\cdot 2257+444\\
 2257 &= 5\cdot 444+37\\
 444 &= 12\cdot 37+0.
\end{align*}
Therefore $\gcd(2701,2257)=37$. We determine $x$ and $y:$
\begin{align*}
37&= 2257-5\cdot 444\\
  &= 2257-5(2701-2257)=-5\cdot 2701+6\cdot 2257.
\end{align*}
We have that 
$$
-5\cdot 2701+6\cdot 2257=\gcd(2701,2257)=37.
$$

(d) The summary of the computations:
\begin{align*}
3397 &= 1\cdot 1849+1548\\
 1849 &= 1\cdot 1548+301\\
 1548 &= 5\cdot 301+43\\
 301 &= 7\cdot 43+0.
\end{align*}
That is, $\gcd(3397,1849)=43$. It remains to compute $x$ and $y:$
\begin{align*}
43&=1548-5\cdot 301\\
  &=1548-5(1849-1548)=-5\cdot 1849+6\cdot 1548\\
  &= -5\cdot 1849+6(3397-1849)=6\cdot 3397-11\cdot 1849.
\end{align*}
Thus we obtain the equation
$$
6\cdot 3397-11\cdot 1849=\gcd(3397,1849)=43.
$$



\item[\ref{ex:numsyst1}]

Write 21 in base 2 first. 
Now, 16 is the highest 2-power not greater than 21, 
$21 = 1 \cdot 16 +5$, 
and we continue with the remainder 5. 
Now, 4 is the highest 2-power not greater than 5, 
$5 = 1\cdot 4 + 1$, 
and we continue with the remainder 1. 
Finally, 1 is the highest 2-power not greater than 1, 
$1 = 1 \cdot 1 + 0$. 
Thus 
\[
21_{10} = 1 \cdot 16 + 1 \cdot 4 + 1\cdot 1 = 1 \cdot 2^4 + 1 \cdot 2^2 + 1 \cdot 2^0 = 10101_2. 
\]

Now, write 50 in base 3. 
Here, 27 is the highest 3-power not greater than 50, 
$50 = 1 \cdot 27 + 23$, 
and we continue with the remainder 23. 
Now, 9 is the highest 3-power not greater than 23, 
$23 = 2\cdot 9 + 5$, 
and we continue with the remainder 5. 
Now, 3 is the highest 3-power not greater than 5, 
$5 = 1\cdot 3 + 2$, 
and we continue with the remainder 2. 
Finally, 1 is the highest 3-power not greater than 2, 
$2 = 2 \cdot 1 + 0$. 
Thus 
\[
50_{10} = 1 \cdot 27 + 2 \cdot 9 + 1 \cdot 3 + 2\cdot 1 = 1 \cdot 3^3 + 2 \cdot 3^2 + 1 \cdot 3^1 + 2 \cdot 3^0  = 1212_3. 
\]

Finally, write 2814 in base 16. 
Now, 256 is the highest 16-power not greater than 2814 
(the next 16-power is 4096), 
$2814 = 10 \cdot 256 + 254$, 
and we continue with the remainder 254. 
Now, 16 is the highest 16-power not greater than 254, 
$254 = 15\cdot 16 + 14$, 
and we continue with the remainder 14. 
Finally, 1 is the highest 16-power not greater than 14, 
$14 = 14 \cdot 1 + 0$. 
Thus 
\[
2814_{10} = 10 \cdot 256 + 15 \cdot 16 + 14 \cdot 1 = 10 \cdot 16^2 + 15 \cdot 16^1 + 14 \cdot 16^0   = AFE_{16}. 
\]


\item[\ref{ex:numsyst2}]

Rewrite $21_{10}$ into base 2 first. 
\begin{align*}
21 &= 10 \cdot 2 + 1, \\
10 &= 5 \cdot 2 + 0, \\
5 &= 2 \cdot 2 + 1, \\
2 &= 1 \cdot 2 + 0, \\
1 &= 0 \cdot 2 + 1. 
\end{align*}
The remainders backwards are 1, 0, 1, 0, 1, thus 
\[
21_{10} = 10101_{2}. 
\]

Now, rewrite $50_{10}$ into base 3. 
\begin{align*}
50 &= 16 \cdot 3 + 2, \\
16 &= 5 \cdot 3 + 1, \\
5 &= 1 \cdot 3 + 2, \\
1 &= 0 \cdot 3 + 1. 
\end{align*}
The remainders backwards are 1, 2, 1, 2, thus 
\[
50_{10} = 1212_{3}. 
\]

Finally, rewrite $250_{10}$ into base 8. 
\begin{align*}
250 &= 31 \cdot 8 + 2, \\
31 &= 3 \cdot 8 + 7, \\
3 &= 0 \cdot 8 + 3. 
\end{align*}
The remainders backwards are 3, 7, 2, thus 
\[
250_{10} = 372_{8}. 
\]

\item[\ref{ex:numsyst3}]
\begin{enumerate}
\item
\begin{align*}
111001101_2 &= 461_{10}, \\
1010101_2 &= 85_{10}, \\
11111_2 &= 31_{10}, \\
10110_2 &= 22_{10}, \\
101010101_2 &= 341_{10}, \\
10001000_2 &= 136_{10}, \\
1010111_2 &= 87_{10}, \\
111101_2 &= 61_{10}, \\
21102_3 &= 200_{10}, \\
1234_5 &= 194_{10}, \\
1234_7 &= 466_{10}, \\
1234_8 &= 668_{10}, \\
777_8 &= 511_{10}, \\
345_8 &= 229_{10}, \\ 
2012_8 &= 1034_{10}, \\
4565_8 &= 2421_{10}, \\
1123_8 &= 595_{10}, \\
666_8 &= 438_{10}, \\
741_8 &= 481_{10}, \\
CAB_{16} &= 3243_{10}, \\ 
BEE_{16} &= 3054_{10}, \\
EEE_{16} &= 3822_{10}, \\
4D4_{16} &= 1236_{10}, \\
ABC_{16} &= 2748_{10}, \\
9B5_{16} &= 2485_{10}, \\
DDD_{16} &=3549_{10}, \\
3F2_{16} &= 1010_{10}.
\end{align*}

\item
\begin{align*}
64_{10} &= 100 0000_2 = 2101_3 = 224_5 = 121_7 = 100_8 = 71_9 = 40_{16}, \\
50_{10} &= 11 0010_2 = 1212_3 = 200_5 = 101_7 = 62_8 = 55_9 = 32_{16}, \\
16_{10} &= 1 0000_2 = 121_3 = 31_5 = 22_7 = 20_8 = 17_9 = 10_{16}, \\
100_{10} &= 110 0100_2 = 1 0201_3 = 400_5 = 202_7 = 144_8 = 121_9 \\
&= 64_{16}, \\
2012_{10} &= 111 1101 1100_2 = 220 2112_3 = 3 1022_5 = 5603_7 = 3734_8 \\
&= 2675_9 = 7DC_{16}, \\
200_{10} &= 1100 1000_2 = 2 1102_3 = 1300_5 = 404_7 = 310_8 = 242_9 \\
&= C8_{16}, \\
151_{10} &= 1001 0111_2 = 1 2121_3 = 1101_5 = 304_7 = 227_8 = 177_9 \\
&= 97_{16}, \\
48_{10} &= 11 0000_2 = 1210_3 = 143_5 = 66_7 = 60_8 = 53_9 = 30_{16}, \\
99_{10} &= 110 0011_2 = 1 0200_3 = 344_5 = 201_7 = 143_8 = 120_9 \\
&= 63_{16}, \\
999_{10} &= 11 1110 0111_2 = 110 1000_3 = 1 2444_5 = 2625_7 = 1747_8 \\
&= 1330_9 = 3E7_{16}. 
\end{align*}

\item
\begin{align*}
1121_3 &= 43_{10} = 101011_2, \\
4312_5 &= 582_{10} = 1461_7, \\
654_8 &= 428_{10} = 525_9, \\
AD2_{16} &= 2770_{10} = 11035_7, \\
543_8 &= 355_{10} = 111011_3, \\
543_9 &= 444_{10} = 121110_3. 
\end{align*}

\item
\begin{align*}
777_8 &= 111111111_2 = 1FF_{16}, \\
345_8 &= 11100101_2 = E5_{16}, \\
2012_8 &= 10000001010_2 = 40A_{16}, \\
456_8 &= 100101110_2 = 12E_{16}, \\
235_8 &= 10011101_2 = 9D_{16}, \\
147_8 &= 1100111_2 = 67_{16}, \\
741_8 &= 111100001_2 = 1E1_{16}, \\ 
CAB_{16} &= 110010101011_2 = 6253_8, \\
BEE_{16} &= 101111101110_2 = 5756_8, \\
EEE_{16} &= 111011101110_2 = 7356_8, \\
4D3_{16} &= 10011010011_2 = 2323_8, \\
ABC_{16} &= 101010111100_2 = 5274_8, \\
FEE_{16} &= 111111101110_2 = 7756_8, \\
9B5_{16} &= 100110110101_2 = 4665_8, \\
3F2_{16} &= 1111110010_2 = 1762_8. 
\end{align*}

\end{enumerate}


\end{enumerate}
\newpage
\section{Counting}

\begin{enumerate}

\item[\ref{ex:sumk}]
Let $n$ be odd first, 
like it was with $n=199$. 
Then if we rearrange the summands (first with last, second with one but last, etc.). 
then the middle term will remain, 
which is $\frac{n+1}{2}$: 
\begin{align*}
& 1 + 2 + \dots + \left(n-1\right) + n = \left(1 + n\right) + \left(2 + n-1\right) + \dots \\
&+ \left(\frac{n-1}{2} + \frac{n+3}{2} \right) + \frac{n+1}{2} = (n+1) + (n+1) + \dots \\
&+ (n+1) +  \frac{n+1}{2} = \left( n+1 \right) \cdot \frac{n-1}{2} + \frac{n+1}{2} \\
&= \left( n+1 \right) \cdot \left( \frac{n-1}{2} + \frac{1}{2} \right) = \frac{(n+1) \cdot n}{2}. 
\end{align*}
If $n$ is even, 
then after rearranging the summands, no term will remain: 
\begin{align*}
& 1 + 2 + \dots + \left(n-1\right) + n = \left(1 + n\right) + \left(2 + n-1\right) + \dots \\
&+ \left(\frac{n}{2} + \frac{n+2}{2} \right) = (n+1) + (n+1) + \dots \\
&+ (n+1) = \left( n+1 \right) \cdot \frac{n}{2} = \frac{(n+1) \cdot n}{2}. 
\end{align*}

\item[\ref{ex:5people3hanshake}]
If everyone shakes hands with three other, 
then they do not shake hand with exactly one person. 
It is easier to consider who does not shake hand with whom. 
The first person does not shake hand with someone. 
Then of the remaining three people the first does not shake hand with someone from these three. 
That leaves one person, who does not shake hand with someone else, 
but everybody else has already been accounted for about not shaking hands with somebody. 
Thus, it is not possible that each of the five people shake hands with three others. 

This argument does not work if someone is allowed to shake hands with someone else more than once. 
Nevertheless, the answer is still \emph{no}. 
Use the same argument we used for proving Corollary~\ref{cor:handshakes}. 
If we sum up all the handshakes for everyone, we obtain $5 \cdot 3 = 15$, 
as each of the 5 people shakes hand with 3 others. 
This way, we counted every handshake twice, 
thus to obtain the number of handshakes we need to divide it by 2. 
But $15/2$ is not an integer, 
while the number of handshakes should be an integer. 
This contradiction proves that it is not possible that each of 5 people shakes hand with 3 others. 

For 7 people we can use this argument, again. 
If we sum up all the handshakes for everyone, we obtain $7 \cdot 3 = 21$, 
as each of the 7 people shakes hand with 3 others. 
This way, we counted every handshake twice, 
thus to obtain the number of handshakes we need to divide it by 2. 
But $21/2$ is not an integer, 
while the number of handshakes should be an integer. 
This contradiction proves that it is not possible that each of 7 people shakes hand with 3 others. 

\item[\ref{ex:kisses}]
The four boys shake hands with each other, 
that is, $\frac{4 \cdot 3}{2} = 6$ handshakes. 
The four girls kisses each other, 
those are $\frac{4 \cdot 3}{2} = 6$ kisses by the same formula we use for handshakes. 
Finally, a boy and a girl kisses, as well. 
All four boys kiss all four girls on the cheek, 
which is $4 \cdot 4 = 16$ more kisses. 
Ultimately, there are 6 handshakes and 22 kisses. 

\item[\ref{ex:isitpossible1}]
\begin{enumerate}
\item
Not possible. 
If there are five packs, each of them containing odd many rabbits, 
then altogether in the five packs there are odd many rabbits 
(odd$+$odd$+$odd$+$odd$+$odd is odd). 
As 100 is not an odd number, 
it is not possible to do the required distribution. 

\item%[\ref{ex:isitpossible2}]
It is possible, 
e.g.\ $3 \cdot 3 \cdot 1 \cdot 1\cdot 1$. 
Another possibility could be $9 \cdot 1 \cdot (-1) \cdot 1 \cdot (-1)$, 
or simply $9$ (as only one integer).

\item%[\ref{ex:isitpossible3}]
It is possible, 
e.g.\  $3 \cdot 3 \cdot 1 \cdot 1\cdot 1\cdot 1 \cdot (-1) \cdot 1 \cdot (-1)$, 
or another possibility is $9 \cdot 1 \cdot (-1) \cdot 1 \cdot (-1) \cdot 1 \cdot (-1) \cdot 1 \cdot (-1)$. 

\item%[\ref{ex:isitpossible4}]
Not possible. 
If the product of integer numbers is 9, then all of them are odd. 
But then the sum of 9 odd integer numbers is odd again, 
and hence cannot be 0. 
\end{enumerate}

\item[\ref{ex:sum24}]
\begin{enumerate}
\item
We can apply Proposition~\ref{prop:sumk} and obtain
\[
1 + 2 + 3 + \dots + 23 + 24 = \frac{24\cdot 25}{2} = 300. 
\]

\item%[\ref{ex:sum24_2}]
This is a bit more tricky, %than Exercise~\ref{ex:sum24_1}, 
but not much. 
One needs to calculate the denominator, as we just calculated the numerator. 
Now, 
\begin{align*}
1-2+3-4+ \dots + 23-24 &= (1-2) + (3-4) + \dots + (23-24) \\
(-1) + (-1) + \dots + (-1) &= -12. 
\end{align*}
Thus the fraction we needed to compute is $\frac{300}{-12} = -25$. 

Another way to calculate the denominator could have been the following: 
\begin{align*}
&1-2+3-4+ \dots + 23-24 = 1+2 + 3+4 + \dots + 23+24 \\
&- 2 \cdot (2 + 4 + \dots + 24) = 300 - 2 \cdot 2 \cdot (1 + 2 + \dots + 12) \\
&= 300 - 4 \cdot \frac{12\cdot 13}{2} = 300- 312 = -12. 
\end{align*}
\end{enumerate}

%\item[\ref{ex:wedding}]
%\begin{enumerate}
%\item
%There are four different ways to choose the wedding couple: 
%that both the bride and the groom has a sibling, 
%that only the bride has a sibling, 
%only the groom has a sibling,
%or neither of them has a sibling. 
%In the first case we have 8 possibilities to choose the registrar, 
%\item
%Yes, it does not matter in which order they choose. 
%\end{enumerate}
%

\item[\ref{ex:noofbase2numbers}]
There is only one possibility for the first digit (it cannot be 0, only 1), 
and there are two possibilities for every other digit. 
Thus, the number of $n$-digit positive integers in base 2 is
\[
1 \cdot \underbrace{2 \cdot \dots \cdot 2}_{n-1} = 2^{n-1}. 
\]


\item[\ref{ex:palindrome}]
If $abc_{10}$ is a base 10 three-digit palindrome number, 
then $a = c$, and $a \neq 0$. 
Thus we can choose $a$ in 9-many ways and $b$ in 10-many ways, 
and hence the number of three-digit palindrome numbers is $9 \cdot 10 = 90$. 

For determining the at most three-digit palindrome numbers, 
we need to find the one-digit long and two-digit long palindrome numbers. 
Every one-digit number is a palindrome number. 
There are exactly 9 two-digit palindrome numbers: 
the $aa_{10}$ numbers for $a\neq 0$. 
That is, 
there are $9+9+90=108$ at most three-digit palindrome numbers. 

Now, consider the number of $n$-digit palindrome numbers in base $k$.
One thing to note is that the first half of the number determines the back half completely. 
Thus we need to count how many ways can we choose the first half. 
Let $n$ be even first. 
Then the first digit is the same as the last digit and differs from 0: 
there are $(k-1)$-many possibilities to choose for the first digit. 
The second digit is the same as the one but last: 
there are $k$-possibilities to choose this digit, etc. 
Finally, the digit at the $n/2$ position is the same as the digit at the $n/2+1$ position:
there are $k$ possibilities to choose this digit. 
Thus, altogether the number of $n$-digit base $k$ palindrome numbers (for even $n$) is 
\[
(k-1) \cdot \underbrace{k \cdot \dots \cdot k}_{n/2-1} = (k-1) \cdot k^{n/2-1}. 
\]
If $n$ is odd, 
then the same argument works, except that the middle digit will not have a pair. 
Thus, altogether the number of $n$-digit base $k$ palindrome numbers (for odd $n$) is 
\[
(k-1) \cdot \underbrace{k \cdot \dots \cdot k}_{(n-1)/2} = (k-1) \cdot k^{(n-1)/2}. 
\]

\item[\ref{ex:Hungarianwords}]
There are 44 letters in the Hungarian alphabet, 
therefore there are $44^{n}$-many $n$ letter long words in Hungarian by Theorem~\ref{thm:sequence}. 
That is, $44^5 , 44^7, 44^{10}$-many %\hspace{10pt}
$5, 7, 10$ letter long words can be created, respectively.

\item[\ref{ex:toto}]
There are three possibilities for every game, 
there are 14 games, 
thus the number of required tickets is
\[
\underbrace{3 \cdot 3 \cdot \dots \cdot 3}_{14} = 3^{14} = 4~782~969. 
\]

\item[\ref{ex:company}]
We apply Theorem~\ref{thm:sequence}. 
Now, we allow spaces, 
thus the alphabet contains 27 letters. 
There are $27^{20}$ possibilities for a 20 letter long string (name), 
2 possibilities for the gender, 
$27^{10}$ possibilities for a 10 letter long string (job title), 
and $10^8$ possibilities for an at most 8 digit long base 10 number (payment). 
Thus, the number of possibilities is
\[
27^{20} \cdot 2 \cdot 27^{10} \cdot 10^{8}. 
\]


\item[\ref{ex:subsetsof3elemetset}]
The subsets of $\halmaz{1, 2, 3}$ are 
$\halmaz{} = \emptyset$, 
$\halmaz{1}$, $\halmaz{2}$, $\halmaz{3}$, 
$\halmaz{1, 2}$, $\halmaz{1, 3}$, $\halmaz{2, 3}$, 
$\halmaz{1, 2, 3}$. 

The subsets of $\halmaz{a, b, c}$ are 
$\halmaz{} = \emptyset$, 
$\halmaz{a}$, $\halmaz{b}$, $\halmaz{c}$, 
$\halmaz{a, b}$, $\halmaz{a, c}$, $\halmaz{b, c}$, 
$\halmaz{a, b, c}$. 

The subsets of $\halmaz{\text{Alice, Beth, Carrie}}$ are 
$\halmaz{} = \emptyset$, 
$\halmaz{\text{Alice}}$, $\halmaz{\text{Beth}}$, $\halmaz{\text{Carrie}}$, 
$\halmaz{\text{Alice, Beth}}$, $\halmaz{\text{Alice, Carrie}}$, $\halmaz{\text{Beth, Carrie}}$, 
and finally $\halmaz{\text{Alice, Beth, Carrie}}$. 

The subsets of $\halmaz{apple, banana, cherry}$ are 
$\halmaz{} = \emptyset$, 
$\halmaz{\text{apple}}$, $\halmaz{\text{banana}}$, $\halmaz{\text{cherry}}$, 
$\halmaz{\text{apple, banana}}$, $\halmaz{\text{apple, cherry}}$, $\halmaz{\text{banana, cherry}}$, 
and $\halmaz{\text{apple, banana, cherry}}$. 

All sets have 8 subsets. 

\item[\ref{ex:abcde}]
The set $\halmaz{a, b, c, d}$ has 16 subsets: 
$\halmaz{} = \emptyset$, 
$\halmaz{a}$, $\halmaz{b}$, $\halmaz{c}$, $\halmaz{d}$, 
$\halmaz{a, b}$, $\halmaz{a, c}$, $\halmaz{a, d}$, $\halmaz{b, c}$, $\halmaz{b, d}$, $\halmaz{c, d}$, 
$\halmaz{a, b, c}$, $\halmaz{a, b, d}$, $\halmaz{a, c, d}$, $\halmaz{b, c, d}$, 
$\halmaz{a, b, c, d}$. 

The set $\halmaz{a, b, c, d, e}$ has 32 subsets: 
$\halmaz{} = \emptyset$, 
$\halmaz{a}$, $\halmaz{b}$, $\halmaz{c}$, $\halmaz{d}$, $\halmaz{e}$, 
$\halmaz{a, b}$, $\halmaz{a, c}$, $\halmaz{a, d}$, $\halmaz{a, e}$, $\halmaz{b, c}$, 
$\halmaz{b, d}$, $\halmaz{b, e}$, $\halmaz{c, d}$, $\halmaz{c, e}$, $\halmaz{d, e}$, 
$\halmaz{a, b, c}$, $\halmaz{a, b, d}$, $\halmaz{a, b, e}$, $\halmaz{a, c, d}$, $\halmaz{a, c, e}$,
$\halmaz{a, d, e}$, $\halmaz{b, c, d}$, $\halmaz{b, c, e}$, $\halmaz{b, d, e}$, $\halmaz{c, d, e}$,
$\halmaz{a, b, c, d}$, $\halmaz{a, b, c, e}$, $\halmaz{a, b, d, e}$, $\halmaz{a, c, d, e}$, $\halmaz{b, c, d, e}$,  
$\halmaz{a, b, c, d, e}$. 



\item[\ref{ex:subsettrytheproof}]
The decision algorithm is collected in 
the following table ($T$ is the subset of $S = \halmaz{a, b, c}$): 
%Table~\ref{tab:subsettrytheproof}. 

%\begin{table}[!htb]
%\caption{Decision algorithm on the subsets of $\halmaz{a, b, c}$.}\label{tab:subsettrytheproof}
%\begin{center}
\begin{tabular}{|c|c|c|c|c|c|c|c|}
\hline
\multicolumn{4}{|c|}{$a \in T$} & \multicolumn{4}{|c|}{$a \notin T$} \\
\hline
\multicolumn{2}{|c|}{$b \in T$} & \multicolumn{2}{|c|}{$b \notin T$} & \multicolumn{2}{|c|}{$b \in T$} & \multicolumn{2}{|c|}{$b \notin T$}\\
\hline
$c \in T$ & $c \notin T$ & $c \in T$ & $c \notin T$ & $c \in T$ & $c \notin T$ & $c \in T$ & $c \notin T$ \\
\hline
$\halmaz{a, b, c}$ & $\halmaz{a, b}$ & $\halmaz{a, c}$ & $\halmaz{a}$ & $\halmaz{b, c}$ & $\halmaz{b}$ & $\halmaz{c}$ & $\halmaz{} = \emptyset $ \\
\hline
\end{tabular}
%\end{center}
%\end{table}

First we decide whether or not $a \in T$, 
then (independently on our first choice) we decide whether or not $b \in T$,
then (independently on our earlier choices) we decide whether or not $c \in T$.
That way, we obtain $2 \cdot 2 \cdot 2 = 8$ subsets. 

\item[\ref{ex:subsetabcd}]
There are 8 subsets of $\halmaz{a, b, c, d}$ not containing $d$: 
$\halmaz{} = \emptyset$, 
$\halmaz{a}$, $\halmaz{b}$, $\halmaz{c}$, 
$\halmaz{a, b}$, $\halmaz{a, c}$, $\halmaz{b, c}$, 
$\halmaz{a, b, c}$. 
These are the subsets of $\halmaz{a, b, c}$. 
There are 8 subsets of $\halmaz{a, b, c, d}$ containing $d$: 
$\halmaz{d}$, $\halmaz{a, d}$, $\halmaz{b, d}$, $\halmaz{c, d}$, 
$\halmaz{a, b, d}$, $\halmaz{a, c, d}$, $\halmaz{b, c, d}$, 
$\halmaz{a, b, c, d}$. 
These are the subsets of $\halmaz{a, b, c}$ with the element $d$ added to them. 


\item[\ref{ex:subsetabcd01}]
The binary representation of the subsets of $\halmaz{a, b, c, d}$ can be seen in Table~\ref{tab:abcdbinary} on page~\pageref{tab:abcdbinary}. 
\begin{table}[!htb]
\caption{Subsets of $\halmaz{a, b, c, d}$ represented as binary numbers}\label{tab:abcdbinary}
\begin{center}
\begin{tabular}{c|c|c}
subset of $\halmaz{a, b, c, d}$ & binary number & decimal number \\
\hline
$\halmaz{}$ & $0000_2$ & 0 \\
$\halmaz{a}$ & $0001_2$ & 1 \\
$\halmaz{b}$ & $0010_2$ & 2 \\
$\halmaz{a, b}$ & $0011_2$ & 3 \\
$\halmaz{c}$ & $0100_2$ & 4 \\
$\halmaz{a, c}$ & $0101_2$ & 5 \\
$\halmaz{b, c}$ & $0110_2$ & 6 \\
$\halmaz{a, b,c }$ & $0111_2$ & 7 \\
$\halmaz{d}$ & $1000_2$ & 8 \\
$\halmaz{a, d}$ & $1001_2$ & 9 \\
$\halmaz{b, d}$ & $1010_2$ & 10 \\
$\halmaz{a, b, d}$ & $1011_2$ & 11 \\
$\halmaz{c, d}$ & $1100_2$ & 12 \\
$\halmaz{a, c, d}$ & $1101_2$ & 13 \\
$\halmaz{b, c, d}$ & $1110_2$ & 14 \\
$\halmaz{a, b, c, d}$ & $1111_2$ & 15 
\end{tabular}
\end{center}
\end{table}

\item[\ref{ex:encode1}]
After computing the binary representation, 
we just add the elements corresponding to the places where the digits are 1. 
\begin{center}
\begin{tabular}{c|c|c}
decimal number & binary number & subset of $S$ \\
\hline 
11 & $1011_2$ & $\halmaz{a_0, a_1, a_3}$ \\
7 & $0111_2$ & $\halmaz{a_0, a_1, a_2}$ \\
15 & $1111_2$ & $\halmaz{a_0, a_1, a_2, a_3}$
\end{tabular}
\end{center}

\item[\ref{ex:encode2}]
After computing the binary representation, 
we just add the elements corresponding to the places where the digits are 1. 
\begin{center}
\begin{tabular}{c|c|c}
decimal number & binary number & subset of $S$ \\
\hline 
11 & $01011_2$ & $\halmaz{a_0, a_1, a_3}$ \\
7 & $00111_2$ & $\halmaz{a_0, a_1, a_2}$ \\
15 & $01111_2$ & $\halmaz{a_0, a_1, a_2, a_3}$ \\
16 & $10000_2$ & $\halmaz{a_4}$ \\
31 & $11111_2$ & $\halmaz{a_0, a_1, a_2, a_3, a_4}$
\end{tabular}
\end{center}
Note, that the encoding was defined in such a way, 
that the subset of $\halmaz{a_0, a_1, a_2, a_3}$ corresponding to $k$ is the same as 
the subset of $\halmaz{a_0, a_1, a_2, a_3, a_4}$ corresponding to $k$ (for arbitrary $0\leq k\leq 15$). 

\item[\ref{ex:encode3}]
After computing the binary representation, 
we just add the elements corresponding to the places where the digits are 1. 
\begin{center}
\begin{tabular}{c|c|c}
decimal number & binary number & subset of $S$ \\
\hline 
49 & $110001_2$ & $\halmaz{a_0, a_4, a_5}$
\end{tabular}
\end{center}

\item[\ref{ex:encode4}]
After computing the binary representation, 
we just add the elements corresponding to the places where the digits are 1. 
\begin{center}
\begin{tabular}{c|c|c}
decimal number & binary number & subset of $S$ \\
\hline 
101 & $1100101_2$ & $\halmaz{a_0, a_2, a_5, a_6}$
\end{tabular}
\end{center}

\item[\ref{ex:encode5}]
After computing the binary representation, 
we just add the elements corresponding to the places where the digits are 1. 
\begin{center}
\begin{tabular}{c|c|c}
decimal number & binary number & subset of $S$ \\
\hline 
199 & $11000111_2$ & $\halmaz{a_0, a_1, a_2, a_6, a_7}$
\end{tabular}
\end{center}


\item[\ref{ex:ExamEFGH}]
All possibilities are listed in Table~\ref{tab:ExamEFGH} on page~\pageref{tab:ExamEFGH}. 

\begin{table}[!htb]
\caption{The orders in which Ed, Frank, George and Hugo can take the exam}\label{tab:ExamEFGH}
\begin{center}
\begin{tabular}{c|c|c|c}
first & second & third & fourth\\
\hline
Ed & Frank & George & Hugo \\
Ed & Frank & Hugo & George \\
Ed & George & Frank & Hugo \\
Ed & George & Hugo & Frank \\
Ed & Hugo & Frank & George \\
Ed & Hugo & George & Frank \\
Frank & Ed & George & Hugo \\
Frank & Ed & Hugo & George \\
Frank & George & Ed & Hugo \\
Frank & George & Hugo & Ed \\
Frank & Hugo & Ed & George \\
Frank & Hugo & George & Ed \\
George & Ed & Frank & Hugo \\
George & Ed & Hugo & Frank \\
George & Frank & Ed & Hugo \\
George & Frank & Hugo & Ed \\
George & Hugo & Ed & Frank \\
George & Hugo & Frank & Ed \\
Hugo & Ed & Frank & George \\
Hugo & Ed & George & Frank \\
Hugo & Frank & Ed & George \\
Hugo & Frank & George & Ed \\
Hugo & George& Ed & Frank \\
Hugo & George & Frank & Ed 
\end{tabular}
\end{center}
\end{table}

\item[\ref{ex:perm1}]
The number of permutations of $\halmaz{1, 2, 3, 4}$ is $4! =24$. 

\item[\ref{ex:perm2}]
The number of permutations of $\halmaz{a, b, c, d}$ is $4! =24$. 

\item[\ref{ex:perm3}]
The number of permutations of 8 people is $8! = 40~320$. 
If the boys sit on seats from 1 to 5, 
and girls sit on seats from 6 to 8, 
then we need to count the number of permutations of the boys and girls separately. 
The boys can sit on their seats in $5! = 120$-many ways. 
The girls (independently on how the boys sit) can sit on their seats in $3! = 6$-many ways. 
Altogether, 
they can sit in $6 \cdot 120 = 720$-many ways. 

\item[\ref{ex:retinas}]
The number of anagrams of `retinas' is the same as the number of permutations of the letters `r', `e', `t', `i', `n', `a' and `s'. 
There are 7 different letters, hence the number of permutations is $7! = 5~040$. 

\item[\ref{ex:puppy}]
Again, let us color the `p's in the anagrams by three colors: 
\textcolor{red}{red}, \textcolor{green}{green}, \textcolor{blue}{blue}. 
This way, there will be $5!=120$-many coloured anagrams of puppy, 
the same as the number of permutations of five different elements. 
Now, group together those anagrams, 
which only differ by their colouring. 
For example the group `puppy' would contain 
`\textcolor{red}{p}u\textcolor{green}{p}\textcolor{blue}{p}y', 
`\textcolor{red}{p}u\textcolor{blue}{p}\textcolor{green}{p}y', 
`\textcolor{green}{p}u\textcolor{red}{p}\textcolor{blue}{p}y', 
`\textcolor{green}{p}u\textcolor{blue}{p}\textcolor{red}{p}y', 
`\textcolor{blue}{p}u\textcolor{red}{p}\textcolor{green}{p}y', 
`\textcolor{blue}{p}u\textcolor{green}{p}\textcolor{red}{p}y'. 
How do we know that there are six coloured `puppy's? 
The coloured `puppy's only differ in the colourings of the `p's. 
The first `p' can be coloured by 3 different colours, 
the next `p' (right after the `u') can be coloured by two different colours 
(it cannot be coloured by the same colour as the first `p'), 
then the last `p' should be coloured by the remaining colour. 
Thus, there are $3 \cdot 2 \cdot 1 = 6$-many coloured `puppy's. 
Similarly, there are 6 coloured versions of every anagram. 
Therefore there are $\frac{120}{6} = 20$ (uncoloured) anagrams of `puppy'. 
These are 
`pppuy',
`pppyu',
`ppupy',
`ppypu',
`ppuyp',
`ppyup',
`puppy',
`pyppu',
`pupyp',
`pypup',
`puypp',
`pyupp',
`upppy',
`ypppu',
`uppyp',
`yppup',
`upypp',
`ypupp',
`uyppp',
`yuppp'. 

\item[\ref{ex:anagrams1}]
\begin{enumerate}
\item
The word `college' contains 7 letters, two of them are `e's and two of them are `l's, 
thus the number of anagrams is 
\[
\frac{7!}{2! \cdot 2!} = \frac{5~040}{2 \cdot 2} = 1~260. 
\]

\item%[\ref{ex:anagrams2}]
The word `discrete' contains 8 letters, two of them are `e's, 
thus the number of anagrams is 
\[
\frac{8!}{2!} = \frac{40~320}{2} = 20~160. 
\]

\item%[\ref{ex:anagrams3}]
The word `mathematics' contains 11 letters, two of them are `a's, two of them are `m's  and two of them are `t's, 
thus the number of anagrams is 
\[
\frac{11!}{2! \cdot 2! \cdot 2!} = \frac{39~916~800}{2 \cdot 2 \cdot 2} = 4~989~600. 
\]

\item%[\ref{ex:anagrams4}]
The expression `discrete mathematics' contains 19 letters, 
two of them are `i's, 
two of them are `s's, 
two of them are `c's, 
three of them are `e's, 
three of them are `t's, 
two of them are `m's  and 
two of them are `a's,  
thus the number of anagrams is 
\begin{align*}
\frac{19!}{2! \cdot 2! \cdot 2! \cdot 3! \cdot 3! \cdot 2! \cdot 2!} &= \frac{121~645~100~408~832~000}{2 \cdot 2 \cdot 2 \cdot 6 \cdot 6 \cdot 2 \cdot 2} \\
&= 105~594~705~216~000. 
\end{align*}

\item%[\ref{ex:anagrams5}]
The expression `college discrete mathematics' contains 26 letters, 
three of them are `c's, 
two of them are `l's, 
five of them are `e's, 
two of them are `i's, 
two of them are `s's, 
three of them are `t's, 
two of them are `m's  and 
two of them are `a's,  
thus the number of anagrams is 
\begin{align*}
\frac{26!}{3! \cdot 2! \cdot 5! \cdot 2! \cdot 2! \cdot 3! \cdot 2! \cdot 2!} &= \frac{403~291~461~126~605~635~584~000~000}{6 \cdot 2 \cdot 120 \cdot 2 \cdot 2 \cdot 6 \cdot 2 \cdot 2} \\
&= 2~917~328~277~825~561~600~000. 
\end{align*}
\end{enumerate}

\item[\ref{ex:anagrams2}]
\emph{First solution.}
Let us create the (not meaningful) word `aaaaabbbbccc', 
and consider its anagrams. 
Put the bouquets into one particular order, 
and consider an anagram. 
This anagram represents a distribution of the bouquets among the triplets: 
if a letter is `a' in the anagram, the corresponding bouquet will be taken by Alice, 
if a letter is `b' in the anagram, the corresponding bouquet will be taken by Beth, 
if a letter is `c' in the anagram, the corresponding bouquet will be taken by Carrie. 
For example, the distribution for the anagram `abcbbaaaccba' is that 
Alice takes the first, sixth, seventh, eighth and twelfth bouquets, 
Beth takes the second, fourth, fifth and eleventh bouquets, 
and Carrie takes the third, ninth and tenth bouquets. 
This gives a one-to-one correspondence between the possible distributions of the bouquets and the anagrams of `aaaaabbbbccc'. 
Therefore by Theorem~\ref{thm:permrepetition} the number of distributions is
\[
\frac{12!}{5! \cdot 4! \cdot 3!} = \frac{479~001~600}{120 \cdot 24 \cdot 6} = 27~720. 
\]

\emph{Second solution.}
Imagine that the triplets put the 12 bouquets in some order, 
and then Alice takes the first 5, 
Beth takes the next four, 
and Carrie takes the last three. 
Thus, the original order of the bouquets determine who gets which bouquet. 
Of course, 
some of these orders give the same result: 
if we only permute the first five elements or the next four elements, 
or the final three elements, 
then clearly everyone obtains exactly the same bouquets. 
Thus, the number of possible distributions is the number of permutations of the 12 bouquets, 
divided by the number of permutations of the first five elements, 
the number of permutations of the next four elements, 
the number of permutations of the last three elements. 
That is, 
the number of possible distributions is 
\[
\frac{12!}{5! \cdot 4! \cdot 3!} = \frac{479~001~600}{120 \cdot 24 \cdot 6} = 27~720. 
\]


\item[\ref{ex:22-6}]
The two numbers are equal, as the following calculation shows
\[
\frac{22!}{16!} = \frac{22 \cdot 21 \cdot 20 \cdot 19 \cdot 18 \cdot 17 \cdot 16!}{16!} 
= 22 \cdot 21 \cdot 20 \cdot 19 \cdot 18 \cdot 17. 
\]

\item[\ref{ex:orderedsubsets}]
Altogether there are $n!$ possible orders for the $n$ elements (this is the number of permutations of $n$ elements). 
But not all of these are considered to be different, because we are only interested in the first $k$ elements. 
Those cases will be considered the same where the first $k$ elements are the same (and in the same order). 
%Just as we did in Section~\ref{sec:anagrams} for counting the anagrams, 
That is, we group together those permutations of the $n$ elements, 
where the order of the first $k$ elements is the same. 
We can name every group with the order of the first $k$ elements. 
Thus, we are interested in the number of groups we have. 
In one group there are those permutations, 
where the order of the first $k$ elements is the same, 
thus they only differ in the last $(n-k)$ elements. 
There are $(n-k)!$ possible permutations of the last $(n-k)$ elements, 
therefore every group contains $(n-k)!$ orderings of the $n$ elements. 
Hence, 
the number of ordered $k$-element subsets is
\begin{align*}
\frac{n!}{(n-k)!} &= \frac{n \cdot (n-1) \cdot \dots \cdot (n-k+1) \cdot (n-k)!}{(n-k)!} \\
&= n \cdot (n-1) \cdot \dots \cdot (n-k+1).
\end{align*}

\item[\ref{ex:forma1}]
By Theorem~\ref{thm:orderedsubsets} the number of possibilities 
\begin{enumerate}
\item
for the first eight cars is
\[
22 \cdot 21 \cdot 20 \cdot 19 \cdot 18 \cdot 17 \cdot 16 \cdot 15 = 12~893~126~400,
\]

\item
for the first ten cars is
\[
22 \cdot 21 \cdot 20 \cdot 19 \cdot 18 \cdot 17 \cdot 16 \cdot 15 \cdot 14 \cdot 13 = 2~346~549~004~800.
\]
\end{enumerate}

\item[\ref{ex:competition}]
By Theorem~\ref{thm:orderedsubsets} the number of ordered subsets is
\begin{enumerate}
\item
\[
10 \cdot 9 \cdot 8 = 720, 
\]

\item
\[
12 \cdot 11 \cdot 10 = 1~320, 
\]

\item
\[
10 \cdot 9 \cdot 8 \cdot 7 = 5~040, 
\]

\item
\[
12 \cdot 11 \cdot 10 \cdot 9 = 11~880, 
\]

\item
\[
8 \cdot 7 \cdot 6 \cdot 5 \cdot 4 = 6~720, 
\]

\item
\[
10 \cdot 9 \cdot 8 \cdot 7 \cdot 6= 30~240. 
\]
\end{enumerate}


\item[\ref{ex:90choose5}]
The two numbers are equal, as the following calculation shows
\[
\frac{90!}{5! \cdot 85!} = \frac{90 \cdot 89 \cdot 88 \cdot 87 \cdot 86 \cdot 85!}{5! \cdot 85!} = \frac{90 \cdot 89 \cdot 88 \cdot 87 \cdot 86}{5!}.
\]


\item[\ref{ex:smallnchoosek}]
The required binomial coefficients are computed and arranged into a triangle in Table~\ref{tab:pascal6} on page~\pageref{tab:pascal6}. 
\begin{table}[!htb]
\caption{Small binomial coefficients.}\label{tab:pascal6}
\begin{center}
\begin{sideways}%table}[!htb]%[H]
%\caption{Small binomial coefficients}\label{tab:pascal6}
\begin{tabular}{cccccccccccccc} 
%$n=0$:
& & & & & & $\binom{0}{0} = 1$ \\
\noalign{\smallskip\smallskip} 
%$n=1$:
& & & & & $\binom{1}{0} = 1$ & & $\binom{1}{1} = 1$ \\
\noalign{\smallskip\smallskip} 
%$n=2$:
& & & & $\binom{2}{0} = 1$ & & $\binom{2}{1} = 2$ & & $\binom{2}{2} = 1$ \\
\noalign{\smallskip\smallskip} 
%$n=3$:
& & & $\binom{3}{0} = 1$ & & $\binom{3}{1} = 3$ & & $\binom{3}{2} = 3$ & & $\binom{3}{3} = 1$ \\
\noalign{\smallskip\smallskip} 
%$n=4$:
& & $\binom{4}{0} = 1$ & & $\binom{4}{1} = 4$ & & $\binom{4}{2} = 6$ & & $\binom{4}{3} = 4$ & & $\binom{4}{4} = 1$ \\
\noalign{\smallskip\smallskip} 
& $\binom{5}{0} = 1$ & & $\binom{5}{1} = 5$ & & $\binom{5}{2} = 10$ & & $\binom{5}{3} = 10$ & & $\binom{5}{4} = 5$ & & $\binom{5}{5} = 1$ \\
\noalign{\smallskip\smallskip} 
$\binom{6}{0} = 1$ & & $\binom{6}{1} = 6$ & & $\binom{6}{2} = 15$ & & $\binom{6}{3} = 20$ & & $\binom{6}{4} = 15$ & & $\binom{6}{5} = 6$ & & $\binom{6}{6} = 1$ \\
\noalign{\smallskip\smallskip} 
\end{tabular}
\end{sideways}%table}
\end{center}
\end{table}

\item[\ref{ex:nicenchoosek}]
By the definition
\begin{align*}
\binom{n}{0} &= \frac{n!}{0! \cdot (n-0)!} = \frac{n!}{0! \cdot n!} = \frac{1}{0!} = \frac{1}{1} = 1, \\
\binom{n}{1} &= \frac{n!}{1! \cdot (n-1)!} = \frac{n \cdot (n-1)!}{1! \cdot (n-1)!} = \frac{n}{1!} = \frac{n}{1} = n, \\
\binom{n}{2} &= \frac{n!}{2! \cdot (n-2)!} = \frac{n \cdot (n-1) \cdot (n-2)!}{2! \cdot (n-2)!} = \frac{n \cdot (n-1)}{2!} = \frac{n \cdot (n-1)}{2}, \\
\binom{n}{n-2} &= \frac{n!}{(n-2)! \cdot (n-(n-2))!} = \frac{n \cdot (n-1) \cdot (n-2)!}{(n-2)! \cdot 2!} = \frac{n \cdot (n-1)}{2!} \\
&= \frac{n \cdot (n-1)}{2}, \\
\binom{n}{n-1} &= \frac{n!}{(n-1)! \cdot (n-(n-1))!} = \frac{n \cdot (n-1)!}{(n-1)! \cdot 1!} = \frac{n}{1!} = \frac{n}{1} = n, \\
\binom{n}{n} &= \frac{n!}{n! \cdot (n-n)!} = \frac{n!}{n! \cdot 0!} = \frac{1}{0!} = \frac{1}{1} = 1.
\end{align*}


\item[\ref{ex:sumsmallnchoosek}]
Using Table~\ref{tab:pascal6} from Exercise~\ref{ex:smallnchoosek}, 
it is not hard to determine the required sums:
\begin{align*}
\sum_{k=0}^0 \binom{0}{k} &= \binom{0}{0} = 1, \\
\sum_{k=0}^1 \binom{1}{k} &= \binom{1}{0} + \binom{1}{1} = 1 + 1 = 2, \\
\sum_{k=0}^2 \binom{2}{k} &= \binom{2}{0} + \binom{2}{1} + \binom{2}{2} = 1 + 2 + 1 = 4, \\
\sum_{k=0}^3 \binom{3}{k} &= \binom{3}{0} + \binom{3}{1} + \binom{3}{2} + \binom{3}{3} = 1 + 3 + 3 + 1 = 8, \\ 
\sum_{k=0}^4 \binom{4}{k} &= \binom{4}{0} + \binom{4}{1} + \binom{4}{2} + \binom{4}{3} + \binom{4}{4} \\
&= 1 + 4 + 6 + 4 + 1 = 16, \\ 
\sum_{k=0}^5 \binom{5}{k} &= \binom{5}{0} + \binom{5}{1} + \binom{5}{2} + \binom{5}{3} + \binom{5}{4} + \binom{5}{5} \\
&= 1 + 5 + 10 + 10 + 5 + 1 = 32, \\
\sum_{k=0}^6 \binom{6}{k} &= \binom{6}{0} + \binom{6}{1} + \binom{6}{2} + \binom{6}{3} + \binom{6}{4} + \binom{6}{5} + \binom{6}{6} \\
&= 1 + 6 + 15 + 20 + 15 + 6 + 1 = 64. 
\end{align*}



\item[\ref{ex:nmidnchoose2}]
Now, $n$ divides $\binom{n}{2}$ if and only if the quotient $\binom{n}{2} / n$ is an integer. 
Here
\[
\frac{\binom{n}{2}}{n} = \frac{\frac{n\cdot (n-1)}{2}}{n} = \frac{n-1}{2}, 
\]
and this is an integer number if and only if $2 \nmid n$, that is, if and only if $n$ is odd. 

\item[\ref{ex:n^2binom}]
Using the formula for $\binom{n+1}{2}$ and $\binom{n}{2}$ we have 
\begin{align*}
\binom{n+1}{2} + \binom{n}{2} &= \frac{(n+1) \cdot n}{2} + \frac{n \cdot (n-1)}{2} = \frac{n^2+n}{2} + \frac{n^2-n}{2} \\
&= \frac{n^2 + n + n^2 -n }{2} = \frac{2n^2}{2} = n^2. 
\end{align*}









\item[\ref{ex:piratesgeq1}]
Let the pirates be $P_1, \dots , P_n$. 
They put the gold pieces in a line. 
Then they want to divide it into $n$ parts by putting sticks between gold pieces. 
The leftmost part will go to $P_1$, 
the next part from left goes to $P_2$, etc. 
The rightmost part will go to $P_n$. 
To this end, they use $n-1$ sticks to divide the $k$ gold pieces into $n$ parts. 
What is left from the first stick is for $P_1$, 
what is between the first and second sticks is for $P_2$, etc., 
and everything right from the last stick is taken by $P_n$. 
They can put the sticks between gold pieces. 
They cannot put a stick before the first gold piece, 
because then $P_1$ would not get any pieces. 
Similarly, they cannot put a stick after the last gold piece, 
because $P_n$ needs to receive at least one gold piece. 
Finally, they cannot put two sticks between the same two gold pieces, 
because then one of the pirates would not get any gold piece.
Thus, they need to put $n-1$ sticks somewhere in the spaces between the gold pieces, 
but they cannot put two sticks between the same two gold pieces. 
That is, they need to find which $n-1$ places they put sticks to. 
There are $k-1$ places between $k$ gold pieces, 
and they need to find $n-1$, where they put the $n-1$ sticks. 
This can be done in $\binom{k-1}{n-1}$-many ways. 

\item[\ref{ex:piratesgeq2}]
Let every pirate take one gold piece right at the very beginning. 
Then there remains $k-n$ gold pieces to further distribute. 
Moreover, now every pirate needs one more gold piece. 
Thus we reduced the problem to the one solved in Theorem~\ref{thm:piratesgeq1}: 
$n$ pirates want to distribute $k-n$ gold pieces such that everyone gets at least one gold piece. 
By Theorem~\ref{thm:piratesgeq1} this can be done in $\binom{k-n-1}{n-1}$-many ways. 

\item[\ref{ex:piratesgeq012}]
All 15 possibilities are written in Table~\ref{tab:piratesgeq012} on page~\pageref{tab:piratesgeq012}. 
\begin{table}[!htb]
\caption{Possibilities to distribute seven gold pieces among three pirates such that Black Bellamy gets at least one gold piece and Calico Jack gets at least two gold pieces.}\label{tab:piratesgeq012}
\begin{center}
\begin{tabular}{|c|c|c|}
\hline
Anne Bonney & Black Bellamy & Calico Jack \\
\hline
\hline
4 & 1 & 2 \\
0 & 5 & 2 \\
0 & 1 & 6 \\
3 & 2 & 2 \\
3 & 1 & 3 \\
1 & 4 & 2 \\
0 & 4 & 3 \\
1 & 1 & 5 \\
0 & 2 & 5 \\
2 & 3 & 2 \\
2 & 1 & 4 \\
0 & 3 & 4 \\
2 & 2 & 3 \\
1 & 3 & 3 \\
1 & 2 & 4 \\
\hline
\end{tabular}
\end{center}
\end{table}

\item[\ref{ex:pirates1}]
By applying Theorem~\ref{thm:pirates}, 
we obtain
\begin{enumerate}
\item
\[
\binom{9 - 1}{3 - 1} = \binom{8}{2} = 28, 
\]

\item
\[
\binom{8+3 - 1}{3 - 1} = \binom{10}{2} = 45, 
\]

\item
\[
\binom{7+3 - 1}{3 - 1} = \binom{9}{2} = 36, 
\]

\item
\[
\binom{11-3 - 1}{3 - 1} = \binom{7}{2} = 21, 
\]

\item
\[
\binom{9 - 1}{4 - 1} = \binom{8}{3} = 56, 
\]

\item
\[
\binom{7+4 - 1}{4 - 1} = \binom{10}{3} = 120, 
\]

\item
\[
\binom{12-4 - 1}{4 - 1} = \binom{7}{3} = 35, 
\]

\item
\[
\binom{10 -1 -2 -3 +4 - 1}{4 - 1} = \binom{7}{3} = 35, 
\]

\item
\[
\binom{15 -1 -2 -3 -4+4 - 1}{4 - 1} = \binom{8}{3} = 56, 
\]

\item
\[
\binom{15 -1 -1 -3 -3+5 - 1}{5 - 1} = \binom{11}{4} = 330. 
\]
\end{enumerate}

\item[\ref{ex:equations1}]
By applying Corollary~\ref{cor:pirates}, 
we obtain that the number of solutions is
\begin{enumerate}
\item
\[
\binom{9 - 1}{3 - 1} = \binom{8}{2} = 28, 
\]

\item
\[
\binom{8+3 - 1}{3 - 1} = \binom{10}{2} = 45, 
\]

\item
\[
\binom{7+3 - 1}{3 - 1} = \binom{9}{2} = 36, 
\]

\item
\[
\binom{11-3 - 1}{3 - 1} = \binom{7}{2} = 21, 
\]

\item
\[
\binom{9 - 1}{4 - 1} = \binom{8}{3} = 56, 
\]

\item
\[
\binom{7+4 - 1}{4 - 1} = \binom{10}{3} = 120, 
\]

\item
\[
\binom{12-4 - 1}{4 - 1} = \binom{7}{3} = 35, 
\]

\item
\[
\binom{10 -1 -2 -3 +4 - 1}{4 - 1} = \binom{7}{3} = 35, 
\]

\item
\[
\binom{15 -1 -2 -3 -4+4 - 1}{4 - 1} = \binom{8}{3} = 56, 
\]

\item
\[
\binom{15 -1 -1 -3 -3+5 - 1}{5 - 1} = \binom{11}{4} = 330. 
\]
\end{enumerate}

\item[\ref{ex:TuroRudi}]
It is the same problem as the gold distribution: 
imagine that everybody of the three siblings gets the brand they like. 
Then the problem is equivalent to distributing 10 desserts among the three children such that everyone gets at least one. 
There are $\binom{10-1}{3-1} = \binom{9}{2} = 45$-many ways to do this by Theorem~\ref{thm:pirates}. 

\item[\ref{ex:ballsinurn1}]
Applying Table~\ref{tab:nchoosek} we obtain that the number of choices is 
\begin{enumerate}
\item
\[
\binom{9 + 3 -1}{3} = \binom{11}{3} = 165, 
\]

\item
\[
\binom{9 + 3 -1}{9} = \binom{11}{9} = 55, 
\]

\item
\[
\frac{10!}{5!} = 30~240, 
\]

\item
\[0,\]

\item
\[
\binom{45}{6} = 8~145~060, 
\]

\item
\[0,\]

\item
\[
100^{10} = 10^{20}, 
\]

\item
\[10^{100}.\] 
\end{enumerate}
\end{enumerate}


\newpage
\section{Proof Techniques}
%\renewcommand{\theenumi}{3.\arabic{enumi}}
\begin{enumerate}
\item[\ref{induction-1}] The statement $S(n)$ is that 8 divides $9^n-1$. Clearly we have
\begin{align*}
9^1-1=8&=1\cdot 8,\\
9^2-1=80&=10\cdot 8.
\end{align*}
Hence $S(1)$ is true and $S(2)$ is true as well. Assume that $S(k)$ is true for some
$k\in\mathbb{N}$. It remains to prove that $S(k+1)$ is true. We have that $S(k)$ is true, that is, 8 divides $9^k-1$. Hence there exists
an integer $A$ such that $9^k-1=8\cdot A$. It remains to prove that $9^{k+1}-1$ is a multiple of 8. We have that
$$
9(9^k-1)=8\cdot A\cdot 9.
$$
Hence we get
$$
9^{k+1}-1=8\cdot A\cdot 9+8=8(9A+1).
$$
That is, 8 divides $9^{k+1}-1$. Thus $S(k+1)$ is true, so the statement is true for all positive integers.

\item[\ref{induction-2}] The statement $S(n)$ is that 6 divides $5^{2n-1}+1$. We compute $5^{2n-1}+1$ for some
small values:
\begin{align*}
5^{2\cdot 1-1}+1=6&=1\cdot 6,\\
5^{2\cdot 2-1}+1=126&=21\cdot 6.
\end{align*}
It is now obvious that $S(1)$ is true and $S(2)$ is true, too. Assume that $S(k)$ is true for some
$k\in\mathbb{N}$. That is, there exists $A$ such that 
$$5^{2k-1}+1=6\cdot A.$$ 
We multiply this latter 
equation by $5^2:$
$$
5^2\cdot 5^{2k-1}+5^2=6\cdot A \cdot 5^2.
$$
We would like to have the expression of $S(k+1)$ on the left-hand side, that is, $5^{2(k+1)-1}+1=5^{2k+1}+1$.
So we subtract 24 to obtain
$$
5^{2k+1}+1=6\cdot A \cdot 5^2-24=6(25A-4).
$$
It follows that 6 divides $5^{2k+1}+1$, hence $S(k+1)$ is true. We have proved that $S(n)$ is true for
all positive integers.

\item[\ref{induction-3}] Here we deal with the sum of the first $n$ odd integers. For $n\in\halmaz{1,2,3,4,5}$
we have
\begin{center}
\begin{tabular}{|c|c|}
\hline
$n$ & $\sum_{i=1}^n (2i-1)$\\
\hline
1 & $1=1^2$\\
\hline
2 & $1+3=2^2$\\
\hline
3 & $1+3+5=3^2$\\
\hline
4 & $1+3+5+7=4^2$\\
\hline
5 & $1+3+5+7+9=5^2$\\
\hline
\end{tabular}
\end{center}
Hence the given formula provides correct answers. Let $S(n)$ be the statement that the sum of the first $n$
odd integers is $n^2$. We have already proved that $S(1)$ is true. Assume that $S(k)$ is true for some
$k\geq 1$, that is,
$$
\sum_{i=1}^k(2i-1)=k^2.
$$
It remains to show that $S(k+1)$ is true, that is,
$$
\sum_{i=1}^{k+1}(2i-1)=(k+1)^2.
$$
The left-hand side can be written as
$$
\sum_{i=1}^{k+1}(2i-1)=\left(1+3+\ldots+(2k-1)\right)+(2k+1).
$$
By the induction hypotheses we have
$$
\left(1+3+\ldots+(2k-1)\right)=k^2,
$$
so we obtain
$$
\left(1+3+\ldots+(2k-1)\right)+(2k+1)=k^2+2k+1=(k+1)^2.
$$
Thus the statement $S(k+1)$ is true and the result follows.

\item[\ref{induction-4}] We consider here the sum of the first $n$ squares, which is
$$
1^2+2^2+\ldots+n^2.
$$
The statement $S(n)$ is that 
$$
\sum_{i=1}^n i^2=\frac{n(n+1)(2n+1)}{6}.
$$
The statement is clearly true for $n=1$, since $1^2=\frac{1\cdot 2\cdot 3}{6}$. Assume that the statement is true
for certain $k\geq 1$, that is, 
$$
\sum_{i=1}^k i^2=\frac{k(k+1)(2k+1)}{6}.
$$
Let us study $S(k+1)$. The sum of the first $k+1$ squares can be written as the sum of the first $k$ squares increased by $(k+1)^2$,
that is, we have
$$
\sum_{i=1}^{k+1} i^2=\left(\sum_{i=1}^k i^2\right)+(k+1)^2.
$$
The induction hypotheses says that 
$$
\sum_{i=1}^k i^2=\frac{k(k+1)(2k+1)}{6},
$$
hence we obtain
$$
\sum_{i=1}^{k+1} i^2=\frac{k(k+1)(2k+1)}{6}+(k+1)^2.
$$
The right-hand side equals to
\begin{align*}
&\frac{k(k+1)(2k+1)}{6}+\frac{6(k+1)^2}{6}=\frac{(k+1)\left(k(2k+1)+6(k+1)\right)}{6}=\\
&=\frac{(k+1)(2k^2+7k+6)}{6}=\frac{(k+1)(k+2)(2k+3)}{6}.
\end{align*}
Therefore $S(k+1)$ is true and the problem has been solved.

\item[\ref{induction-4a}] We list the sum of the first $n$ cubes in the following table for $n\in\halmaz{1,2,3,4}$. 
\begin{center}
\begin{tabular}{|c|c|}
\hline
$n$ & $\sum_{i=1}^n i^3$\\
\hline
1 & $1^3=1=\left(\frac{1\cdot 2}{2}\right)^2$\\
\hline
2 & $1^3+2^3=9=\left(\frac{2\cdot 3}{2}\right)^2$\\
\hline
3 & $1^3+2^3+3^3=36=\left(\frac{3\cdot 4}{2}\right)^2$\\
\hline
4 & $1^3+2^3+3^3+4^3=100=\left(\frac{4\cdot 5}{2}\right)^2$\\
\hline
\end{tabular}
\end{center}
The statement $S(n)$ to prove is that 
$$
\sum_{i=1}^n i^3=\left(\frac{n(n+1)}{2}\right)^2.
$$
We showed that $S(1),S(2),S(3)$ and $S(4)$ are true. Assume that for some $1\leq k\in\mathbb{N}$ the statement $S(k)$ is true. We try to conclude that $S(k+1)$ is true.
We have 
$$
\sum_{i=1}^{k+1} i^3= \left(\sum_{i=1}^{k} i^3\right)+(k+1)^3.
$$
By the induction hypothesis we can write the right-hand side as
\begin{align*}
& \left(\frac{k(k+1)}{2}\right)^2+(k+1)^3=\frac{k^2(k+1)^2+4(k+1)^3}{4}=\\
& \frac{(k+1)^2}{4}(k^2+4k+4)=\left(\frac{(k+1)(k+2)}{2}\right)^2.
\end{align*}
It follows that $S(k+1)$ is true and therefore the identity is true for all positive integers $n$.

\item[\ref{ex:sumk(k+1)}]
Let $S(n)$ be the statement that 
\[
\sum_{i=1}^{n-1}{i(i+1)}=\frac{\left(n-1\right)n\left(n+1\right)}{3}.
\]
If $n=2$, then the left-hand side is
\[
\sum_{i=1}^1{i(i+1)}={2}, 
\]
and the right-hand side is $\frac{1\cdot 2 \cdot 3}{3}$, hence $S(1)$ is true. Assume that $S(k)$ is true for some $1\leq k\in\mathbb{N}$.
The statement $S(k+1)$ says that
\[
\sum_{i=1}^{k}{i(i+1)}=\frac{k\left(k+1\right)\left(k+2\right)}{3}.
\]
On the other hand, by the induction hypothesis
\begin{align*}
& \sum_{i=1}^{k}{i(i+1)}=\left(\sum_{i=1}^{k-1}{i(i+1)}\right)+{k(k+1)}=\\
& =\frac{\left(k-1\right)k\left(k+1\right)}{3}+{k(k+1)}=\frac{\left(k-1\right)k\left(k+1\right)+3k(k+1)}{3}=\\
& =\frac{k\left(k+1\right)\left(k+2\right)}{3}.
\end{align*}
Therefore $S(k+1)$ is true and the identity
\[
\sum_{i=1}^{n-1} {i(i+1)}=\frac{\left(n-1\right)n\left(n+1\right)}{3}
\]
is valid for all positive integers $n$.


\item[\ref{induction-4b}] Let $S(n)$ be the statement that 
$$
\sum_{i=1}^n\frac{1}{i(i+1)}=\frac{n}{n+1}.
$$
If $n=1$, then the left-hand side is
$$
\sum_{i=1}^1\frac{1}{i(i+1)}=\frac{1}{2}, 
$$
and the right-hand side is $\frac{1}{2}$, hence $S(1)$ is true. Assume that $S(k)$ is true for some $1\leq k\in\mathbb{N}$.
The statement $S(k+1)$ says that
$$
\sum_{i=1}^{k+1}\frac{1}{i(i+1)}=\frac{k+1}{k+2}.
$$
On the other hand, by the induction hypothesis
\begin{align*}
& \sum_{i=1}^{k+1}\frac{1}{i(i+1)}=\left(\sum_{i=1}^{k}\frac{1}{i(i+1)}\right)+\frac{1}{(k+1)(k+2)}=\\
& =\frac{k}{k+1}+\frac{1}{(k+1)(k+2)}=\frac{k(k+2)+1}{(k+1)(k+2)}=\\
& =\frac{(k+1)^2}{(k+1)(k+2)}=\frac{k+1}{k+2}.
\end{align*}
Therefore $S(k+1)$ is true and the identity
$$
\sum_{i=1}^n\frac{1}{i(i+1)}=\frac{n}{n+1}
$$
is valid for all positive integers $n$.


\item[\ref{induction-5}] Let us compute the first few elements of the sequence
\begin{center}
\begin{tabular}{|c|c|}
\hline
$n$ & $a_n$\\
\hline
1 & 1\\
\hline
2 & 8\\
\hline
3 & $a_2+2a_1=10$\\
\hline
4 & $a_3+2a_2=26$\\
\hline
5 & $a_4+2a_3=46$\\
\hline
\end{tabular}
\end{center}
Now we compute the values of the formula $\frac{3}{2}\cdot 2^n+2\cdot (-1)^n$ for $n\in\halmaz{1,2,3,4,5}$
\begin{center}
\begin{tabular}{|c|c|}
\hline
$n$ & $\frac{3}{2}2^n+2(-1)^n$\\
\hline
1 & 1\\
\hline
2 & 8\\
\hline
3 & 10\\
\hline
4 & 26\\
\hline
5 & 46\\
\hline
\end{tabular}
\end{center}
We checked that $a_n=\frac{3}{2}\cdot 2^n+2\cdot (-1)^n$ for $n\in\halmaz{1,2,3,4,5}$. Assume that the statement is true
for $S(k-1)$ and $S(k)$ for some $2\leq k\in\mathbb{N}$, that is,
\begin{align*}
a_{k-1}&=\frac{3}{2}\cdot 2^{k-1}+2\cdot (-1)^{k-1},\\
a_k&=\frac{3}{2}\cdot 2^k+2\cdot (-1)^k.
\end{align*}
The statement for $k+1$ is that $a_{k+1}=\frac{3}{2}\cdot 2^{k+1}+2\cdot (-1)^{k+1}$. By definition we have that
$$
a_{k+1}=a_{k}+2a_{k-1}.
$$
Hence by the induction hypotheses we obtain
$$
a_{k+1}=\frac{3}{2}\cdot 2^k+2\cdot (-1)^k+2\left(\frac{3}{2}\cdot 2^{k-1}+2\cdot (-1)^{k-1}\right).
$$
A direct computation yields that
$$
a_{k+1}=\frac{3}{2}\cdot 2^{k+1}+2\cdot (-1)^{k+1},
$$
which is the statement we wanted to prove. We showed that the given formula is correct.

\item[\ref{induction-6}] We note that there is a solution in Chapter~\ref{Recurrence sequences} which does not
use induction. Here we will prove it by induction. Our statement $S(n)$ is that the number 
$$
f(n)=\left(\frac{3-\sqrt{33}}{2}\right)^n+\left(\frac{3+\sqrt{33}}{2}\right)^n
$$
is an integer and it is a multiple of 3. Let us consider the statement for $n=1$ and 2.
If $n=1$, then we get that $f(1)=3$. So it is an integer and it is a multiple of 3. If $n=2$, then we have
$$
f(2)=\frac{9-6\sqrt{33}+33}{4}+\frac{9+6\sqrt{33}+33}{4}=21.
$$
Again we obtained an integer that is divisible by 3. Assume that $S(k-1)$ and $S(k)$ are true for some $2\leq k\in\mathbb{N}$.
We will use the fact that the numbers $\frac{3-\sqrt{33}}{2}$ and $\frac{3+\sqrt{33}}{2}$ are roots of the quadratic equation
$$
x^2-3x-6,
$$
that is, we have
\begin{align*}
\left(\frac{3-\sqrt{33}}{2}\right)^2&= 3 \cdot \left(\frac{3-\sqrt{33}}{2}\right)+6,\\
\left(\frac{3+\sqrt{33}}{2}\right)^2&= 3 \cdot \left(\frac{3+\sqrt{33}}{2}\right)+6.
\end{align*}
The induction hypothesis says that $f(k-1)$ and $f(k)$ are integers which are multiples of 3. What about the number $f(k+1)$?
We have
\begin{align*}
& f(k+1)=\left(\frac{3-\sqrt{33}}{2}\right)^{k+1}+\left(\frac{3+\sqrt{33}}{2}\right)^{k+1}=\\
& \left(\frac{3-\sqrt{33}}{2}\right)^{k-1}\left(3\cdot \left(\frac{3-\sqrt{33}}{2}\right)+6\right)
+\left(\frac{3+\sqrt{33}}{2}\right)^{k-1}\left(3\cdot \left(\frac{3+\sqrt{33}}{2}\right)+6\right)=\\
& 3\cdot \left(\left(\frac{3-\sqrt{33}}{2}\right)^{k}+\left(\frac{3+\sqrt{33}}{2}\right)^{k}\right)+6\cdot \left(\left(\frac{3-\sqrt{33}}{2}\right)^{k-1}+\left(\frac{3+\sqrt{33}}{2}\right)^{k-1}\right)=\\
& 3f(k)+6f(k-1).
\end{align*}
Since $f(k-1)$ and $f(k)$ are integers we see that $f(k+1)$ is an integer and it is a multiple of 3.

\item[\ref{induction-7}] We compute $a_n$ for some $n:$
\begin{center}
\begin{tabular}{|c|c|}
\hline
$n$ & $a_n$\\
\hline
1 & $\approx 1.4142$\\
\hline
2 & $\approx 1.8477$\\
\hline
3 & $\approx 1.9615$\\
\hline
4 & $\approx 1.9903$\\
\hline
5 & $\approx 1.9975$\\
\hline
\end{tabular}
\end{center}
Let $S(n)$ be the statement that $a_n\leq 2$. Our computations show that $S(1)$ is true. Assume that $S(k)$ is true for some $k\geq 1$,
that is, $a_k\leq 2$. We consider the statement $S(k+1)$. By definition of the sequence we have 
$$
a_{k+1}=\sqrt{2+a_k}.
$$
By the induction hypothesis $a_k\leq 2$, hence
$$
a_{k+1}\leq \sqrt{2+2}=2.
$$
The statement $S(k+1)$ has been proved and thus we have that $a_n\leq 2$ for all $n\in\mathbb{N}$.

\item[\ref{induction-8}]
We prove the statement by induction. If $n=1$, then the 1-digit integer $a_1=2$ is divisible by 2. Therefore the statement is true
for $n=1$. It is not difficult to deal with the case $n=2$. There are only four possible integers
$$
a_1a_2\in\halmaz{11,12,21,22}.
$$
It is easy to see that $2^2$ divides 12. Assume that the statement is true for some $1\leq k\in\mathbb{N}$, that is, there exists a
$k$-digit integer $a_1a_2\ldots a_k$ which is a multiple of $2^k$. Let us consider the statement for $k+1$. By induction hypothesis
we have
$$
a_1a_2\ldots a_k=2^k\cdot A.
$$
We claim that either
$$
10^{k}+a_1a_2\ldots a_k=1a_1a_2\ldots a_k
$$
or
$$
2\cdot 10^{k}+a_1a_2\ldots a_k=2a_1a_2\ldots a_k
$$
is a multiple of $2^{k+1}$. We can rewrite the above integers as follows
\begin{align*}
& 10^{k}+a_1a_2\ldots a_k=10^k+2^k\cdot A=2^k(5^k+A),\\
& 2\cdot 10^{k}+a_1a_2\ldots a_k=2\cdot 10^{k}+2^k\cdot A=2^k(2\cdot 5^k+A).
\end{align*}
If $A$ is odd, then $5^k+A$ is even. In this case 
$$
1a_1a_2\ldots a_k
$$
is an integer having $k+1$ digits and it is divisible by $2^{k+1}$.
If $A$ is even, then $2\cdot 5^k+A$ is even. That is, 
$$
2a_1a_2\ldots a_k
$$
is a $(k+1)$-digit number which is a multiple of $2^{k+1}$.

\item[\ref{induction-9}]

(a) The first few elements of the Fibonacci sequence are
$$
1,1,2,3,5,8,13,21,34,55,\ldots
$$
Let us consider the sum of the first $n$ elements for $n\in\halmaz{1,2,3,4,5}$
\begin{align*}
n&=1: 1,\\
n&=2: 1+1=2,\\
n&=3: 1+1+2=4,\\
n&=4: 1+1+2+3=7,\\
n&=5: 1+1+2+3+5=12.
\end{align*}
It is easy to see that the identity holds for $n\in\halmaz{1,2,3,4,5}$. Assume that the statement is true for some $1\leq k\in\mathbb{N}$,
that is, 
$$
F_1+F_2+\ldots+F_k=F_{k+2}-1.
$$
Consider the sum of the first $k+1$ Fibonacci numbers
$$
F_1+F_2+\ldots+F_k+F_{k+1}.
$$
By induction hypothesis we get
$$
F_1+F_2+\ldots+F_k+F_{k+1}=F_{k+2}-1+F_{k+1}.
$$
By definition $F_{k+1}+F_{k+2}=F_{k+3}$, hence
$$
F_1+F_2+\ldots+F_k+F_{k+1}=F_{k+3}-1.
$$
Therefore the identity holds for all positive integers.

(b) Compute $F_1^2+F_2^2+\ldots+F_n^2$ for $n\in\halmaz{1,2,3,4,5}$
\begin{align*}
n&=1: 1^2=1=F_1F_2,\\
n&=2: 1^2+1^2=2=F_2F_3,\\
n&=3: 1^2+1^2+2^2=6=F_3F_4,\\
n&=4: 1^2+1^2+2^2+3^2=15=F_4F_5,\\
n&=5: 1^2+1^2+2^2+3^2+5^2=40=F_5F_6.
\end{align*}
That is, the identity is valid for $n\in\halmaz{1,2,3,4,5}$. Assume that the statement is true for some $1\leq k\in\mathbb{N}$,
that is, 
$$
F_1^2+F_2^2+\ldots+F_k^2=F_kF_{k+1}.
$$
Let us deal with the sum
$$
F_1^2+F_2^2+\ldots+F_k^2+F_{k+1}^2.
$$
Applying the induction hypothesis we obtain
$$
(F_1^2+F_2^2+\ldots+F_k^2)+F_{k+1}^2=F_kF_{k+1}+F_{k+1}^2=F_{k+1}(F_k+F_{k+1}),
$$
and by definition $F_k+F_{k+1}=F_{k+2}$, so we have
$$
(F_1^2+F_2^2+\ldots+F_k^2)+F_{k+1}^2=F_{k+1}F_{k+2}.
$$
Thus the identity holds for all positive integers.

(c) Here we consider the identity
$$
F_1+F_3+\ldots+F_{2n-1}=F_{2n}.
$$
If $n=1$, then the left-hand side is $F_1=1$ and the right-hand side is $F_2=1$. Hence the identity holds.
Assume that for some $1\leq k\in\mathbb{N}$ the identity is valid, that is, 
$$
F_1+F_3+\ldots+F_{2k-1}=F_{2k}.
$$
In case of $k+1$ terms we have
$$
F_1+F_3+\ldots+F_{2k-1}+F_{2k+1}.
$$
By induction we get
$$
(F_1+F_3+\ldots+F_{2k-1})+F_{2k+1}=F_{2k}+F_{2k+1}=F_{2k+2}=F_{2(k+1)}.
$$
Thus the identity is valid for all positive integers.

(d) The identity to prove is as follows: 
$$
F_2+F_4+\ldots+F_{2n}=F_{2n+1}-1.
$$
If $n=1$, then the left-hand side is $F_2=1$ and the right-hand side is $F_3-1=2-1=1$. So the identity is valid.
Assume that for some $1\leq k\in\mathbb{N}$ the identity holds, that is, 
$$
F_2+F_4+\ldots+F_{2k}=F_{2k+1}-1.
$$
Let us handle the sum for $k+1$ terms, that is, the sum
$$
F_2+F_4+\ldots+F_{2k}+F_{2k+2}.
$$
It can be written as 
$$
(F_2+F_4+\ldots+F_{2k})+F_{2k+2}=F_{2k+1}-1+F_{2k+2}=F_{2k+3}-1.
$$
Thus the identity has been proved for all positive integers.

\item[\ref{induction-10}]

(a) First compute $F_{3n}$ for some $n$, let say for $n=1,2,3$.
We have
\begin{align*}
& F_3=2,\\
& F_6=8,\\
& F_9=34.
\end{align*}
We checked that $F_{3n}$ is even for $n=1,2,3$. Assume that $F_{3k}$ is even for some $1\leq k\in\mathbb{N}$. For $k+1$ we
have $F_{3(k+1)}=F_{3k+3}$. By definition
$$
F_{3k+3}=F_{3k+2}+F_{3k+1}=F_{3k+1}+F_{3k}+F_{3k+1}=2\cdot F_{3k+1}+F_{3k}.
$$
By induction $F_{3k}$ is even, so $2\cdot F_{3k+1}+F_{3k}$ is even. The statement is true.

(b) If $n=1$, then $F_{5\cdot 1}=5$. That is, the property holds for $n=1$. Assume that $F_{5k}$ is a multiple of 5 for 
some $1\leq k\in\mathbb{N}$. For $k+1$ we have
\begin{align*}
& F_{5(k+1)}=F_{5k+5}=F_{5k+4}+F_{5k+3}=F_{5k+3}+F_{5k+2}+F_{5k+2}+F_{5k+1}=\\
& 3F_{5k+2}+2F_{5k+1}=3(F_{5k+1}+F_{5k})+2F_{5k+1}=5F_{5k+1}+3F_{5k}.
\end{align*}
It is clear that 5 divides $5F_{5k+1}$ and induction hypothesis implies that $3F_{5k}$ is a multiple of 5.
Therefore 5 divides $5F_{5k+1}+3F_{5k}$. We proved the property for $k+1$. It follows that $F_{5n}$ is a multiple of 
5 for all $n \in \mathbb{N}$.



\item[\ref{contra-0}] We provide an indirect proof. Assume that $x\leq 5$ and $y\leq 5$. We obtain that
$$
x+y\leq 10,
$$
a contradiction, since $x+y>10$, by assumption.

\item[\ref{contra-0a}] Suppose the opposite of the statement, that is, there exists an integer $n$ for which $n^2-2$ is divisible by 4.
Then for some $k\in\mathbb{Z}$ we have $n^2-2=4k$. Hence $n^2=2(2k+1)$. It follows that $n$ is even, so $n=2n_1$ for some integer $n_1$.
We obtain that 
$$
4n_1^2=2(2k+1).
$$
After dividing the equation by 2 we get
$$
2(n_1^2-k)=1,
$$
a contradiction since 2 does not divide 1.

\item[\ref{contra-1}] Assume the opposite, that is, the number $\sqrt{2}+\sqrt{3}$ is rational. Then there exist
$a$ and $b$ such that $\sqrt{2}+\sqrt{3}=\frac{a}{b}$ for some $a\in\mathbb{Z}$ and $b\in\mathbb{N}$ and the greatest common divisor of $a$ and $b$ is 1.
Squaring both sides of the equation $\sqrt{2}+\sqrt{3}=\frac{a}{b}$ we get
$$
2+2\sqrt{6}+3=\frac{a^2}{b^2}.
$$
That is, 
$$
\sqrt{6}=\frac{a^2-5b^2}{2b^2}.
$$
On the right-hand side there is a rational number, so to get a contradiction we have to prove that $\sqrt{6}$ is irrational.
We prove it indirectly. Assume that $\sqrt{6}$ is rational. Then there exist $c$ and $d$ such that $\sqrt{6}=\frac{c}{d}$ for some $c\in\mathbb{Z}$ and $d\in\mathbb{N}$ 
and the greatest common divisor of $c$ and $d$ is 1. We obtain that
$$
6d^2=c^2,
$$
that is, $c$ is even. Therefore $c=2c_1$ for some $c_1$. It follows that
$$
6d^2=4c_1^2\Rightarrow 3d^2=2c_1^2.
$$
We have that 2 divides $3d^2$. Since 2 does not divide 3 it must divide $d^2$. It means that $d^2$ is even, so $d$ is even, a contradiction.
We have proved that $\sqrt{6}$ is irrational and thus we have that $\sqrt{2}+\sqrt{3}$ is irrational.

\item[\ref{contra-2}] Suppose the opposite of the statement, that is, there exists $x=\frac{p}{q}$ with $p\in\mathbb{Z},q\in\mathbb{N}$
and $\gcd(p,q)=1$ such that
$$
a\left(\frac{p}{q}\right)^2+b\frac{p}{q}+c=0.
$$
Multiply the equation by $q^2$ to obtain (we note that $q\neq 0$)
$$
ap^2+bpq+cq^2=0.
$$
Assume that $p$ is even and $q$ is odd. In this case $ap^2$ is even, $bpq$ is even and $cq^2$ is odd. Hence $ap^2+bpq+cq^2$ is odd,
a contradiction. Now assume that $p$ is odd and $q$ is even. Here we obtain that $ap^2$ is odd, $bpq$ is even and $cq^2$ is even.
Again we get a contradiction. Finally, assume that $p$ is odd and $q$ is odd. We get that $ap^2$ is odd, $bpq$ is odd and $cq^2$ is odd,
so $ap^2+bpq+cq^2$ is odd, a contradiction. We remark that the case $p$ is even, $q$ is even is not possible since $\gcd(p,q)=1$.
The statement follows.

\item[\ref{contra-3}] Assume that the statement is false, that is, 
\begin{align*}
a_1&<\frac{a_1+a_2+\ldots+a_n}{n},\\
a_2&<\frac{a_1+a_2+\ldots+a_n}{n},\\
&\vdots\\
a_n&<\frac{a_1+a_2+\ldots+a_n}{n}.
\end{align*}
Take the sum of the above inequalities to get
$$
a_1+a_2+\ldots+a_n<n\cdot\left(\frac{a_1+a_2+\ldots+a_n}{n}\right)=a_1+a_2+\ldots+a_n.
$$
That is, we obtained a contradiction and the statement is proved.

\item[\ref{contra-4}] We provide an indirect proof. Assume that there exists a positive integer $n$ for which
$$
\gcd(F_n,F_{n+1})=d>1.
$$
That is, $d$ divides $F_n$ and $d$ divides $F_{n+1}$. Then $d$ divides the difference of these two Fibonacci
numbers. The difference is equal to
$$
F_{n+1}-F_n=F_{n-1}.
$$
Since $d$ divides the left-hand side we obtain that $d$ divides the right-hand side, that is, $d\mid F_{n-1}$.
We apply the previous argument again
$$
d\mid F_n\mbox{ and }d\mid F_{n-1}\mbox{ hence }d\mid (F_n-F_{n-1}).
$$
In this way we get that $d\mid F_{n-2}$. Now we have that $d\mid F_{n-1}$ and $d\mid F_{n-2}$, so $d\mid (F_{n-1}-F_{n-2})$.
Since $F_{n-1}-F_{n-2}=F_{n-3}$ we obtain that $d$ divides $F_{n-3}$. We continue this process to reach a 
contradiction, namely that $d\mid F_2=1$ and $d\mid F_1=1$. Hence for consecutive Fibonacci numbers $F_n$ and $F_{n+1}$
we have
$$
\gcd(F_n,F_{n+1})=1.
$$

\item[\ref{proof-cons-1}] First we solve the equation $5x_1+7x_2=1$ in integers $x_1,x_2$ by applying
the Euclidean algorithm. We get that 
$$
5\cdot 3+7\cdot(-2)=1,
$$
therefore 
$$
(3n,-2n)
$$
is a solution to the equation $5x_1+7x_2=n$. By Theorem~\ref{LinDioph} we obtain a parametric formula 
$$
(3n-7t,-2n+5t)\quad t\in\mathbb{Z}
$$
for the integer solutions $(x_1,x_2)$ of the equation $5x_1+7x_2=n$. To have non-negative solutions we need
\begin{align*}
3n-7t&\geq 0\Rightarrow t\leq \frac{3n}{7}\\
-2n+5t&\geq 0\Rightarrow t\geq \frac{2n}{5}.
\end{align*}
So we have the following inequalities
$$
\frac{2n}{5}\leq t\leq \frac{3n}{7}.
$$
If there is a $t\in\mathbb{Z}$ in the interval $[\frac{2n}{5},\frac{3n}{7}]$, then $n$ can be represented
in the form $5x_1+7x_2$. Denote by $I_n$ the set $\halmazvonal{t}{\frac{2n}{5}\leq t\leq \frac{3n}{7}, t\in\mathbb{Z}}$. 
\begin{center}
\begin{tabular}{|c|c||c|c||c|c||c|c|}
\hline
$n$ & $I_n$ & $n$ & $I_n$ & $n$ & $I_n$ & $n$ & $I_n$\\
\hline
1 & $\emptyset$ & 8 & $\emptyset$ &15 & $\halmaz{ 6 }$ &22 & $\halmaz{ 9 }$\\
\hline
2 & $\emptyset$ & 9 & $\emptyset$ &16 & $\emptyset$ &23 & $\emptyset$\\
\hline
3 & $\emptyset$ &10 & $\halmaz{ 4 }$ &17 & $\halmaz{ 7 }$ &24 & $\halmaz{ 10 }$\\
\hline
4 & $\emptyset$ &11 & $\emptyset$ &18 & $\emptyset$ &25 & $\halmaz{ 10 }$\\
\hline
5 & $\halmaz{ 2 }$ &12 & $\halmaz{ 5 }$ &19 & $\halmaz{ 8 }$ &26 & $\halmaz{ 11 }$\\
\hline
6 & $\emptyset$ &13 & $\emptyset$ &20 & $\halmaz{ 8 }$ &27 & $\halmaz{ 11 }$\\
\hline
7 & $\halmaz{ 3 }$ &14 & $\halmaz{ 6 }$ &21 & $\halmaz{ 9 }$ &28 & $\halmaz{ 12 }$\\
\hline
\end{tabular}
\end{center}
Hence $n=23$ cannot be represented. However for all $n\in\halmaz{24,25,26,27,28}$ we have 
solutions. From these solutions we can easily obtain solutions for all $n\geq 24$.

\item[\ref{proof-cons-2}] An integer solution to the equation $4x_1+5x_2=1$ is
$$
(x_1,x_2)=(-1,1).
$$
Hence we have a particular solution 
$$
(x_1,x_2)=(-n,n)
$$
in case of the equation $4x_1+5x_2=n$. It is now clear that all solutions can be obtained from the parametrization
$$
(-n-5t,n+4t)\quad t\in\mathbb{Z}.
$$
To have non-negative solutions one needs
\begin{align*}
-n-5t&\geq 0\Rightarrow t\leq \frac{-n}{5}\\
n+4t&\geq 0\Rightarrow t\geq \frac{-n}{4}.
\end{align*}
Thus we have
$$
-\frac{n}{4}\leq t\leq -\frac{n}{5}.
$$
Denote by $I_n$ the set $\halmazvonal{t}{-\frac{n}{4}\leq t\leq -\frac{n}{5}, t\in\mathbb{Z}}$. 
We determine $I_n$ for $1\leq n\leq 15$.
\begin{center}
\begin{tabular}{|c|c||c|c||c|c|}
\hline
$n$ & $I_n$ & $n$ & $I_n$ & $n$ & $I_n$\\
\hline
1 & $\emptyset$ & 6 & $\emptyset$ &11 & $\emptyset$\\
\hline
2 & $\emptyset$ & 7 & $\emptyset$ &12 & $\halmaz{ -3 }$\\
\hline
3 & $\emptyset$ & 8 & $\halmaz{ -2 }$ &13 & $\halmaz{ -3 }$\\
\hline
4 & $\halmaz{ -1 }$ & 9 & $\halmaz{ -2 }$ &14 & $\halmaz{ -3 }$\\
\hline
5 & $\halmaz{ -1 }$ &10 & $\halmaz{ -2 }$ &15 & $\halmaz{ -3 }$\\
\hline
\end{tabular}
\end{center}
So we get that $n=11$ cannot be represented in the appropriate form. 
The solutions for $n\in\halmaz{12,13,14,15}$ can be used to determine solutions for
$n\geq 12$. In case of 13 we have
$$
13=4\cdot 2+5\cdot 1.
$$
It implies that
$$
4(k+3)+1=4(k+2)+5\cdot 1\quad k\in\mathbb{N}\cup\halmaz{0}.
$$
Thus, if $n=4K+1$, then $(x_1,x_2)=(K-1,1),K\geq 1$.

\item[\ref{proof-cons-3}] We can rewrite the equation as 
$$
2(2x_1+3x_2)+9x_3=n
$$
and another possibility is as follows
$$
4x_1+3(2x_2+3x_3)=n.
$$
We will use the second form because the largest coefficient is 4 while in case of the first form the largest
coefficient is 9. So we have
$$
4x_1+3y_1=n.
$$
A particular solution is $(n,-n)$, therefore we get
\begin{align*}
x_1&=n+3t,\\
y_1&=-n-4t
\end{align*}
for some $t\in\mathbb{Z}$. We have that $y_1=2x_2+3x_3=-n-4t$. This equation has a particular solution
$(n-2t,-n)$, therefore we obtain that
\begin{align*}
x_2&=n-2t+3s,\\
x_3&=-n-2s
\end{align*}
for some $s,t\in\mathbb{Z}$. That is, the parametrization of the integral solution of the equation is
given by
\begin{align*}
x_1&=n+3t,\\
x_2&=n-2t+3s,\\
x_3&=-n-2s
\end{align*}
for some $s,t\in\mathbb{Z}$.

\item[\ref{proof-cons-4}] In the previous exercise we determined the parametrization of the integer 
solutions of the equation. It is as follows
\begin{align*}
x_1&=n+3t,\\
x_2&=n-2t+3s,\\
x_3&=-n-2s
\end{align*}
for some $s,t\in\mathbb{Z}$. We would like to have only non-negative integer solutions hence we
get the system of inequalities
\begin{align*}
0&\leq n+3t,\\
0&\leq n-2t+3s,\\
0&\leq -n-2s.
\end{align*}
We easily obtain upper bound for $s$ and lower bound for $t$ as follows
\begin{align*}
-\frac{n}{3}&\leq t,\\
-\frac{n}{2}&\geq s.
\end{align*}
Our system of inequalities implies that
\begin{align*}
-2n\leq &6t\leq 3n+9s,\\
-2n+4t\leq &6s\leq -3n.
\end{align*}
That is, 
\begin{align*}
-5n &\leq 9s,\\
 4t &\leq -n.
\end{align*}
Now we have lower bound for $s$ and upper bound for $t$. Define the intervals $I_s,I_t$ as follows $I_s=[-\frac{5n}{9},-\frac{n}{2}]$
and $I_t=[-\frac{n}{3},-\frac{n}{4}]$. The length of $I_s$ is at least 1 if $n\geq 18$ and the length of $I_t$ is at least 1 if $n\geq 12$.
Therefore the equation has a non-negative integer solution if $n\geq 18$. It remains to handle the cases $1\leq n\leq 17$.
\begin{center}
\begin{tabular}{|c|c|c|c|}
\hline
$n$ & integer(s) in $I_s$ & integer(s) in $I_t$ & solution(s): $(x_1,x_2,x_3)$\\
\hline
1 & - & - & -\\
\hline
2 & -1 & - & -\\
\hline
3 & - & -1 & -\\
\hline
4 & -2 & -1 & $(1,0,0)$\\
\hline
5 & - & - & -\\
\hline
6 & -3 & -2 & $(0,1,0)$\\
\hline
7 & - & -2 & -\\
\hline
8 & -4 & -2 & $(2,0,0)$\\
\hline
9 & -5 & -3 & $(0,0,1)$\\
\hline
10 & -5 & -3 & $(1,1,0)$\\
\hline
11 & -6 & -3 & -\\
\hline
12 & -6 & -4,-3 & $(0,2,0),(3,0,0)$\\
\hline
13 & -7 & -4 & $(1,0,1)$\\
\hline
14 & -7 & -4 & $(2,1,0)$\\
\hline
15 & -8 & -5,-4 & $(0,1,1)$\\
\hline
16 & -8 & -5,-4 & $(1,2,0),(4,0,0)$\\
\hline
17 & -9 & -5 & $(2,0,1)$\\
\hline
\end{tabular}
\end{center}
Thus the largest positive integer $n$ for which the equation 
$$
4x_1+6x_2+9x_3=n
$$
has no non-negative integer solution is 11.


\item[\ref{pigeon-0a}] The pigeonholes are the possible birthdays, there are 366 pigeonholes.
There are 367 people (playing the role of pigeons). Therefore there is at least one pigeonhole
containing at least two people.

\item[\ref{pigeon-0b}] The pigeonholes are the possible birthdays, so there are 366 pigeonholes.
There are 1500 people and $\frac{1500}{366}\approx 4.098$, hence there is at least one pigeonhole containing at least
4 people. That means that at least 4 people were born on the same day of the year.

\item[\ref{pigeon-1}] We define the pigeonholes as disjoint subsets of the set 
$$\halmaz{1,2,3,4,5,6,7,8,9,10,11,12},$$
that is, 
$$
\halmaz{1,12},\halmaz{2,11},\halmaz{3,10},\halmaz{4,9},\halmaz{5,8},\halmaz{6,7}.
$$
The pigeons are the selected integers. We have six pigeonholes and seven pigeons, therefore there exists
a subset containing two selected integers. The subsets are constructed in such a way that the sum of the elements
are 13, so the sum of the two integers belonging to the same subset is 13. 

\item[\ref{pigeon-2}] The pigeons are the 11 chosen integers. We define the pigeonholes as follows: 
\begin{align*}
& \halmaz{1,2},\halmaz{3,4},\halmaz{5,6},\halmaz{7,8},\halmaz{9,10},\\
& \halmaz{11,12},\halmaz{13,14},\halmaz{15,16},\halmaz{17,18},\halmaz{19,20}.
\end{align*}
There are 10 pigeonholes and 11 pigeons, so there exists a pigeonhole containing 2 pigeons.
Thus the statement is true.

\item[\ref{pigeon-3}] We divide the unit square into four smaller squares:

\setlength{\unitlength}{1.5cm}
\begin{center}
\begin{picture}(2, 2)
  \linethickness{0.075mm}
  \multiput(0, 0)(1, 0){3}{\line(0, 1){2}}
  \multiput(0, 0)(0, 1){3}{\line(1, 0){2}}
  \put(.2, .7){\circle*{.04}}
  \put(1.6, 1.38){\circle*{.04}}
  \put(1.77, 1.29){\circle*{.04}}
  \put(.67, 1.35){\circle*{.04}}
  \put(1.23, .33){\circle*{.04}}
\end{picture}
\end{center}
% \begin{center}
% \begin{tabular}{|ccc|ccc|}
% \hline
%  & . &  &  &  & \\
%  &  &  &  & . & \\
%  &  &  &  &  & \\
%  \hline
%  &  &  &  &  & \\
% . &  &  &  &  & .\\
%  &  &  & . &  & \\
% \hline
% \end{tabular}
% \end{center}
The subsquares are the pigeonholes and the points are the pigeons. Hence by the pigeonhole principle
there are at least two points in the same subsquare. The largest distance in a subsquare is the length 
of the diagonal which is $\sqrt{2}/2$. The statement is proved.

\item[\ref{pigeon-4}] We can write the elements of $A$ in the form $2^a\cdot b$, where $a\geq 0$ and 
$b$ is an odd integer between 1 and 99. There are 50 odd integers in the interval $[1,2,\ldots,100]$
therefore the pigeonhole principle implies that among the 51 integers there are at least two with the same
$b$. That is, we have two integers $2^{a_1}\cdot b$ and $2^{a_2}\cdot b$. If $a_1<a_2$, then 
$2^{a_1}\cdot b$ divides $2^{a_2}\cdot b$. The statement is proved.

\item[\ref{pigeon-5}] We apply Theorem~\ref{pigeon-III}. Here the pigeonholes are the grades, so $n=5$.
There are $m_1$ students who get grade 1, $m_2$ students who get grade 2 etc. According to the theorem
one needs $m_1+m_2+m_3+m_4+m_5-5+1$ students to ensure that for some $i$ $m_i$ students get the same grade.
Hence we take $m_1=m_2=m_3=m_4=m_5=4$. Therefore there must be at least 16 students in the class.

\item[\ref{pigeon-6}] We note that it is possible to place 14 bishops such that they cannot hit each other,
a solution is given by
\begin{center}
\begin{tabular}{|c|c|c|c|c|c|c|c|}
\hline
$\circledast$ & $\circledast$ & $\circledast$ & $\circledast$ & $\circledast$ & $\circledast$ & $\circledast$ & $\circledast$\\
 \hline
 &  &  &  &  &  &  & \\
 \hline
 &  &  &  &  &  &  & \\
 \hline
 &  &  &  &  &  &  & \\
 \hline
 &  &  &  &  &  &  & \\
 \hline
 &  &  &  &  &  &  & \\
 \hline
 &  &  &  &  &  &  & \\
 \hline
 & $\circledast$ & $\circledast$ & $\circledast$ & $\circledast$ & $\circledast$ & $\circledast$ & \\
 \hline
\end{tabular}
\end{center}
It remains to show that it is not possible to place more than 14 bishops in such a way that they can not hit each other.
A natural idea is to divide the 64 chess squares into 14 groups such that if two bishops are in the same group then
they can hit each other. We can produce 14 such groups
\begin{center}
\begin{tabular}{|c|c|c|c|c|c|c|c|}
\hline
4 & 8 & 5 & 9 & 6 & 10 & 7 & 11\\
\hline
8 & 4 & 9 & 5 & 10 & 6 & 11 & 7\\
\hline
3 & 9 & 4 & 10 & 5 & 11 & 6 & 12\\
\hline
9 & 3 & 10 & 4 & 11 & 5 & 12 & 6\\
\hline
2 & 10 & 3 & 11 & 4 & 12 & 5 & 13\\
\hline
10 & 2 & 11 & 3 & 12 & 4 & 13 & 5\\
\hline
1 & 11 & 2 & 12 & 3 & 13 & 4 & 14\\
\hline
11 & 1 & 12 & 2 & 13 & 3 & 14 & 4\\
\hline
\end{tabular}
\end{center}

\item[\ref{card-1}]

(a) The first card is $7\clubsuit$, hence the suit of the hidden card is $\clubsuit$. The distance can be obtained from the following table
\begin{center}
\begin{tabular}{|c|c|}
\hline
distance & order of the 3 cards\\
\hline
1 & $3\diamondsuit,J\diamondsuit,A\spadesuit$\\
\hline
2 & $3\diamondsuit,A\spadesuit,J\diamondsuit$\\
\hline
3 & $J\diamondsuit,3\diamondsuit,A\spadesuit$\\
\hline
4 & $J\diamondsuit,A\spadesuit,3\diamondsuit$\\
\hline
5 & $A\spadesuit,3\diamondsuit,J\diamondsuit$\\
\hline
6 & $A\spadesuit,J\diamondsuit,3\diamondsuit$\\
\hline
\end{tabular}
\end{center}
That is, we have 
$$
d(7\clubsuit,\mbox{hidden card})=1.
$$
Thus the hidden card is $8\clubsuit$.

(b) The suit of the secret card is $\diamondsuit$ since the first card in the sequence is $J\diamondsuit$.
It remains to decode the distance of the secret card and $J\diamondsuit:$
\begin{center}
\begin{tabular}{|c|c|}
\hline
distance & order of the 3 cards\\
\hline
1 & $9\clubsuit,8\heartsuit,Q\heartsuit$\\
\hline
2 & $9\clubsuit,Q\heartsuit,8\heartsuit$\\
\hline
3 & $8\heartsuit,9\clubsuit,Q\heartsuit$\\
\hline
4 & $8\heartsuit,Q\heartsuit,9\clubsuit$\\
\hline
5 & $Q\heartsuit,9\clubsuit,8\heartsuit$\\
\hline
6 & $Q\heartsuit,8\heartsuit,9\clubsuit$\\
\hline
\end{tabular}
\end{center}
We get that the distance is 2, so the secret card is $K\diamondsuit$.

(c) It is clear that the suit of the hidden card is $\heartsuit$. We can determine the distance using the following table
\begin{center}
\begin{tabular}{|c|c|}
\hline
distance & order of the 3 cards\\
\hline
1 & $6\diamondsuit,J\diamondsuit,10\heartsuit$\\
\hline
2 & $6\diamondsuit,10\heartsuit,J\diamondsuit$\\
\hline
3 & $J\diamondsuit,6\diamondsuit,10\heartsuit$\\
\hline
4 & $J\diamondsuit,10\heartsuit,6\diamondsuit$\\
\hline
5 & $10\heartsuit,6\diamondsuit,J\diamondsuit$\\
\hline
6 & $10\heartsuit,J\diamondsuit,6\diamondsuit$\\
\hline
\end{tabular}
\end{center}
In this case the distance is 6, therefore the hidden card is $2\heartsuit$.

(d) The first card is $10\diamondsuit$, so the suit of the hidden card is $\diamondsuit$. It remains to figure out the distance.
\begin{center}
\begin{tabular}{|c|c|}
\hline
distance & order of the 3 cards\\
\hline
1 & $5\diamondsuit,2\spadesuit,4\spadesuit$\\
\hline
2 & $5\diamondsuit,4\spadesuit,2\spadesuit$\\
\hline
3 & $2\spadesuit,5\diamondsuit,4\spadesuit$\\
\hline
4 & $2\spadesuit,4\spadesuit,5\diamondsuit$\\
\hline
5 & $4\spadesuit,5\diamondsuit,2\spadesuit$\\
\hline
6 & $4\spadesuit,2\spadesuit,5\diamondsuit$\\
\hline
\end{tabular}
\end{center}
The order of the remaining three cards is $4\spadesuit,2\spadesuit,5\diamondsuit$, hence the distance is 6.
Thus the announced hidden card is $3\diamondsuit$.

(e) It is easy to see that the suit of the hidden card is $\spadesuit$. We can determine the distance using the following table
\begin{center}
\begin{tabular}{|c|c|}
\hline
distance & order of the 3 cards\\
\hline
1 & $7\diamondsuit,3\heartsuit,7\heartsuit$\\
\hline
2 & $7\diamondsuit,7\heartsuit,3\heartsuit$\\
\hline
3 & $3\heartsuit,7\diamondsuit,7\heartsuit$\\
\hline
4 & $3\heartsuit,7\heartsuit,7\diamondsuit$\\
\hline
5 & $7\heartsuit,7\diamondsuit,3\heartsuit$\\
\hline
6 & $7\heartsuit,3\heartsuit,7\diamondsuit$\\
\hline
\end{tabular}
\end{center}
That is, the distance of the two cards is 2. Therefore the hidden card is $10\spadesuit$.

\item[\ref{card-2}]

(a) We need two cards having the same suit. In this example there are two possibilities
\begin{align*}
& 3\clubsuit,7\clubsuit\Rightarrow\mbox{ the hidden card is }7\clubsuit,\mbox{ the distance is 4,}\\
& 2\diamondsuit,5\diamondsuit\Rightarrow\mbox{ the hidden card is }5\diamondsuit, \mbox{ the distance is 3.}
\end{align*}

If the hidden card is $7\clubsuit$, then we have to encode 4 using the cards 
$$
2\diamondsuit<5\diamondsuit<A\spadesuit.
$$
We have that 4 corresponds to the ordering 2,3,1, therefore the order of the remaining four cards is
$$
3\clubsuit,5\diamondsuit,A\spadesuit,2\diamondsuit.
$$

If the secret card is $5\diamondsuit$, then the distance to encode is 3, which is 2,1,3. The correct order of the four
cards is
$$
2\diamondsuit,7\clubsuit,3\clubsuit,A\spadesuit.
$$

(b) The assistant chooses $A\diamondsuit$ as the card to announce. We have that $d(J\diamondsuit,A\diamondsuit)=3$.
One encodes distance 3 as 2,1,3. Thus the order of the four cards is 
$$
J\diamondsuit,8\heartsuit,10\clubsuit,4\spadesuit.
$$

(c) The hidden card is $8\spadesuit$, the distance is 1, therefore the assistant hands the four cards in the following order
to the magician
$$
7\spadesuit,A\clubsuit,6\diamondsuit,K\heartsuit.
$$

(d) Here the assistant may use the following pairs of cards
\begin{align*}
(I). &\ 3\spadesuit,7\spadesuit\Rightarrow\mbox{ hidden card: }7\spadesuit,\mbox{ distance: }4,\\
(II). &\  3\spadesuit,9\spadesuit\Rightarrow\mbox{ hidden card: }9\spadesuit,\mbox{ distance: }6,\\
(III). &\  7\spadesuit,9\spadesuit\Rightarrow\mbox{ hidden card: }9\spadesuit,\mbox{ distance: }2.
\end{align*}
The corresponding sequences are
\begin{align*}
(I). &\ 3\spadesuit,8\diamondsuit,9\spadesuit,7\clubsuit,\\
(II). &\ 3\spadesuit,7\spadesuit,8\diamondsuit,7\clubsuit,\\
(III). &\ 7\spadesuit,7\clubsuit,3\spadesuit,8\diamondsuit.
\end{align*}


(e) In this case the assistant has two choices
\begin{align*}
(I). &\ J\diamondsuit,Q\diamondsuit\Rightarrow\mbox{ hidden card: }Q\diamondsuit,\mbox{ distance: }1,\\
(II). &\ 7\heartsuit,10\heartsuit\Rightarrow\mbox{ hidden card: }10\heartsuit,\mbox{ distance: }3.
\end{align*}
The corresponding sequences of cards are
\begin{align*}
(I). &\ J\diamondsuit,7\heartsuit,10\heartsuit,3\spadesuit,\\
(II). &\ 7\heartsuit,Q\diamondsuit,J\diamondsuit,3\spadesuit.
\end{align*}

\end{enumerate}


\newpage
\section{Pascal's triangle}

%\renewcommand{\theenumi}{4.\arabic{enumi}}

\begin{enumerate}
%\stepcounter{enumi}
\item[\ref{ex:pascalbinomialsame}]
We prove that the two triangles are the same by induction. 
That is, we prove that the $k$th element of the $n$th row is the same in both triangles, 
and we prove this by induction on $n$. 
For $n=0$ the two triangles have the same zero row: $\binom{0}{0} = 1$. 
Assume that the two triangles are equal in the $(n-1)$st row. 
We prove that the two triangles contain the same numbers in the $n$th row, as well. 
The first and the last numbers are the same: $\binom{n}{0} = 1$, $\binom{n}{n} = 1$. 
Now, consider the $k$th element of the $n$th row for an arbitrary $1\leq k\leq n-1$. 
In Pascal's triangle it is the sum of the two numbers above it, 
that is, it is the sum of the $(k-1)$st and the $k$th number of row $(n-1)$. 
By the induction hypotheses, 
row $n-1$ is the same in Pascal's triangle and in the triangle of the Binomial coefficients. 
That is, the $(k-1)$st element of row $(n-1)$ is $\binom{n-1}{k-1}$, 
the $k$th element of row $(n-1)$ is $\binom{n-1}{k}$. 
By Proposition~\ref{prop:binomsum} we have $\binom{n-1}{k-1}+ \binom{n-1}{k} = \binom{n}{k}$. 
That is, the $k$th number of the $n$th row in Pascal's triangle is the same 
as the $k$th number of the $n$th row in the triangle of the Binomial coefficients. 
This is true for arbitrary $1\leq k\leq n-1$, 
thus the $n$th row of Pascal's triangle is the same as the $n$th row of the triangle of the binomial coefficients. 
This finishes the induction proof, 
the two triangles are the same. 

\item[\ref{ex:pascal12}]
The first twelve rows of Pascal's triangle can be seen in Table~\ref{tab:pascal12} on page~\pageref{tab:pascal12}. 

\begin{table}[!htb]
\caption{First twelve rows of Pascal's triangle.}\label{tab:pascal12}
\begin{center}
\begin{sideways}%table}[!htb]%[H]
%\caption{First twelve rows of Pascal's triangle}\label{tab:pascal10}
\begin{tabular}{ccccccccccccccccccccccccc} 
& & & & & & & & & & & & 1\\
\noalign{\smallskip\smallskip} 
& & & & & & & & & & & 1 & & 1\\
\noalign{\smallskip\smallskip} 
& & & & & & & & & & 1 & & 2 & & 1\\
\noalign{\smallskip\smallskip} 
& & & & & & & & & 1 & & 3 & & 3 & & 1\\
\noalign{\smallskip\smallskip} 
& & & & & & & & 1 & & 4 & & 6 & & 4 & & 1\\
\noalign{\smallskip\smallskip} 
& & & & & & & 1 & & 5 & & 10 & & 10 & & 5 & & 1 \\
\noalign{\smallskip\smallskip} 
& & & & & & 1 & & 6 & & 15 & & 20 & & 15 & & 6 & & 1 \\
\noalign{\smallskip\smallskip} 
& & & & & 1 & & 7 & & 21 & & 35 & & 35 & & 21 & & 7 & & 1 \\
\noalign{\smallskip\smallskip} 
& & & & 1 & & 8 & & 28 & & 56 & & 70 & & 56 & & 28 & & 8 & & 1 \\
\noalign{\smallskip\smallskip} 
& & & 1 & & 9 & & 36 & & 84 & & 126 & & 126 & & 84 & & 36 & & 9 & & 1 \\
\noalign{\smallskip\smallskip} 
& & 1 & & 10 & & 45 & & 120 & & 210 & & 252 & & 210 & & 120 & & 45 & & 10 & & 1 \\
\noalign{\smallskip\smallskip} 
& 1 & & 11 & & 55 & & 165 & & 330 & & 462 & & 462 & & 330 & & 165 & & 55 & & 11 & & 1 \\
\noalign{\smallskip\smallskip} 
1 & & 12 & & 66 & & 220 & & 495 & & 792 & & 924 & & 792 & & 495 & & 220 & & 66 & & 12 & & 1 \\
\noalign{\smallskip\smallskip} 
\end{tabular}
\end{sideways}%table}
\end{center}
\end{table}

%\stepcounter{enumi}

\item[\ref{ex:binomialwriteout}]
The Binomial theorem holds for $n=0$ and $n=1$, as well: 
$(x+y)^0 = 1 = \binom{0}{0} x^0 y^0$, 
$(x+y)^1 = x + y = \binom{1}{0} x^1 y^0 + \binom{1}{1} x^0 y^1$. 
Now, we can prove the theorem by induction on $n$. 
Assume that the statement holds for $n-1$, 
that is, 
\[
(x+y)^{n-1} = x^{n-1} + \binom{n-1}{1} x^{n-2}y + \dots + \binom{n-1}{1} xy^{n-2} + y^{n-1}. 
\]
This is the induction hypothesis. 
Now, compute $(x+y)^n$ using the same method as before, 
and use the induction hypothesis for expanding $(x+y)^{n-1}$: 
\begin{align}
\notag (x+y)^n &= (x+y)^{n-1} \cdot (x+y) \\ 
\notag &= \left( x^{n-1} + \binom{n-1}{1} x^{n-2}y + \dots + \binom{n-1}{n-2} xy^{n-2} + y^{n-1} \right) \cdot \left( x + y \right) \\
\notag &= \left( x^{n-1} + \binom{n-1}{1} x^{n-2}y + \dots + \binom{n-1}{n-2} xy^{n-2} + y^{n-1} \right) \cdot x \\
\notag &+ \left( x^{n-1} + \binom{n-1}{1} x^{n-2}y + \dots + \binom{n-1}{n-2} xy^{n-2} + y^{n-1} \right) \cdot y \\
\notag &= x^{n} + \binom{n-1}{1} x^{n-1}y + \dots + \binom{n-1}{n-2} x^2y^{n-2} + xy^{n-1} \\
\notag &+ x^{n-1}y + \binom{n-1}{1} x^{n-2}y^2 + \dots + \binom{n-1}{n-2} xy^{n-1} + y^{n} \\
\notag &= x^{n} + \binom{n-1}{1} x^{n-1}y + \dots + \binom{n-1}{n-2} x^2y^{n-2} + \binom{n-1}{n-1} xy^{n-1} \\
\notag &+ \binom{n-1}{0} x^{n-1}y + \binom{n-1}{1} x^{n-2}y^2 + \dots + \binom{n-1}{n-2} xy^{n-1} + y^{n} \\
\notag &= x^n + \left( \binom{n-1}{1} + \binom{n-1}{0} \right) x^{n-1}y + \dots \\
%\notag &+ \left( \binom{n-1}{2} + \binom{n-1}{1} \right) x^{n-2}y^2 + \dots 
\notag &\dots + \left( \binom{n-1}{k} + \binom{n-1}{k-1} \right) x^{n-k}y^k + \dots \\
\notag &\dots + \left( \binom{n-1}{n-1} + \binom{n-1}{n-2} \right) xy^{n-1} + y^n \\
\label{eq:binomsum2} &= x^n + \binom{n}{1} x^{n-1}y + \dots %\binom{n}{2} x^{n-2}y^2 + \dots \\
%\notag &
+ \binom{n}{k} x^{n-k}y^k + \dots %+ \binom{n}{n-2}x^2y^{n-2} 
+ \binom{n}{n-1}xy^{n-1} + y^n.
\end{align}
Here, we have expanded $(x+y)^{n-1}$ using the induction hypothesis, 
and multiplied it by $(x+y)$ by expanding the brackets. 
Then we collected together the same terms $x^{n-k}y^k$ for every $k=0, 1, \dots , n-1, n$. 
Finally, in \eqref{eq:binomsum2} we used the generating rule of Pascal's triangle (Proposition~\ref{prop:binomsum}). 

\item[\ref{ex:binomialnchoosek}]
Consider $(x+y)^n$: 
\[
(x+y)^n = a_n x^n + a_{n-1}x^{n-1}y + a_{n-2}x^{n-2}y^2 + \dots + a_2x^2y^{n-2} + a_1xy^{n-1} + a_0y^n, 
\] 
for some numbers $a_n, a_{n-1}, \dots, a_1, a_0$. 
How do we obtain the coefficient $a_k$ for $x^{n-k}y^k$? 
Now, $(x+y)^n$ is the $n$-fold product of $(x+y)$ by itself: 
\[
(x+y)^n = \underbrace{(x+y) (x+y)  \dots (x+y) (x+y)}_{n\text{ times}}. 
\]
The multiplication of these $n$ factors is carried out by choosing a term from each factor ($x$ or $y$) in every possible way, 
multiplying these $n$ terms, and then adding the resulting products together. 
Thus the coefficient of $x^{n-k}y^k$ is the number of possibilities to choose $(n-k)$-times the $x$ and $k$-times the $y$ out of the $n$ factors. 
Altogether there are $n$ many $y$'s to choose from, 
and we need to choose $k$ of them (and the remaining $(n-k)$ factors will be chosen as $x$). 
This can be done in $\binom{n}{k}$-many ways. 
%Now, we need to determine which 2 of the 6 y's we choose. 
%This can be done in $\binom{6}{2}$-many ways, 
Therefore the coefficient of $x^{n-k}y^k$ is $a_k=\binom{n}{k}$. 

\item[\ref{ex:binomial1+1}]
\begin{align*}
2^n &= (1+1)^n = 1^n + n \cdot 1^{n-1}\cdot 1 + \binom{n}{2} \cdot 1^{n-2}\cdot 1^2 + \dots + n \cdot 1 \cdot 1^{n-1} + 1^n \\
&= 1 + n + \binom{n}{2} + \dots + \binom{n}{k} + \dots + \binom{n}{n-2} + n + 1 = \sum_{k=0}^n \binom{n}{k}. 
\end{align*}

\item[\ref{ex:binomial1-1}]
\begin{align*}
0 &= 0^n = (1-1)^n \\
&= 1^n + n \cdot 1^{n-1}\cdot (-1) + \binom{n}{2} \cdot 1^{n-2}\cdot (-1)^2 + \dots + n \cdot 1 \cdot (-1)^{n-1} + (-1)^n \\
&= 1 - n  + \binom{n}{2}  - \dots + (-1)^{k} \binom{n}{k} + \dots + (-1)^{n-1} n + (-1)^n \\
&= \sum_{k=0}^n (-1)^k \binom{n}{k}. 
\end{align*}

\item[\ref{ex:expandusingbinomialthm}]
Using the Binomial theorem we obtain  
\begin{align*}
(x+y)^8&= \sum_{k=0}^8 \binom{8}{k} x^{8-k} y^k = x^8 + 8 x^7y + 28x^6y^2 + 56 x^5y^3 \\
&+ 70 x^4 y^4 + 56 x^3 y^5 + 28 x^2 y^6 + 8 x y^7 + y^8, \\
(x-y)^8&= \sum_{k=0}^8 \binom{8}{k} x^{8-k} \left( -y \right)^k = x^8 - 8 x^7y + 28x^6y^2 - 56 x^5y^3 \\
&+ 70 x^4 y^4 - 56 x^3 y^5 + 28 x^2 y^6 - 8 x y^7 + y^8, \\
(a+1)^{10} &= \sum_{k=0}^{10} \binom{10}{k} \cdot a^{10-k}\cdot 1^k = a^{10} + 10a^9 + 45a^8 + 120a^7 \\
&+ 210a^6 + 252a^5 + 210a^4 + 120 a^3 + 45 a^2 + 10a + 1, \\
(b-3)^5 &= \sum_{k=0}^5 \binom{5}{k} b^{5-k} \left( -3 \right)^k = b^5 -15 b^4 + 90 b^3 \\
&-270 b^2 + 405 b - 243, \\
(1 + 2/x)^5 &= \sum_{k=0}^5 \binom{5}{k} \cdot 1^{5-k} \cdot \left( \frac{2}{x} \right)^k = 1 + \frac{10}{x} + \frac{40}{x^2} + \frac{80}{x^3} + \frac{80}{x^4} + \frac{32}{x^5}, \\
\left( a + b \right)^6 &= \sum_{k=0}^6 \binom{6}{k} a^{6-k} b^k = a^6 + 6a^5b + 15 a^4 b^2 + 20 a^3 b^3 \\
&+ 15 a^2 b^4 + 6ab^5 + b^6, \\
\left( 1 + x \right)^5 &= \sum_{k=0}^5 \binom{5}{k} \cdot 1^{5-k} \cdot x^k = 1 + 5x + 10x^2 \\
&+ 10x^3 + 5x^4 + x^5, \\
\left(3a + 4b \right)^4 &= \sum_{k=0}^4 \binom{4}{k} \cdot \left(3a\right)^{4-k} \cdot \left(4b\right)^k = \left( 3a \right)^4 + 4 \cdot \left( 3a \right)^3 \cdot \left( 4b \right) \\
&+ 6 \cdot \left( 3a \right)^2 \cdot \left( 4b \right)^2 + 4 \cdot \left( 3a \right) \cdot \left( 4b \right)^3 + \left( 4b \right)^4  = 81 a^4 \\
&+ 432 a^3b + 864 a^2b^2 + 768 ab^3 + 256 b^4, \\
\left( 3 - 2x \right)^4 &= \sum_{k=0}^4 \binom{4}{k} \cdot 3^{4-k} \cdot \left(-2x\right)^k = 3^4 - 4 \cdot 3^3 \cdot \left( 2x \right) + 6 \cdot 3^2 \cdot \left( 2x \right)^2 \\
&- 4 \cdot 3 \cdot \left( 2x \right)^3 + \left( 2x \right)^4  = 81 - 216 x + 216 x^2 - 96 x^3 + 16 x^4.
\end{align*}

\item[\ref{ex:coefficientinbinomialthm}]
By the Binomial theorem 
\[
\left( 1-\frac{x}{2} \right)^9 = \sum_{k=0}^9 \binom{9}{k} \cdot 1^{9-k} \cdot \left( - \frac{x}{2} \right)^k = \sum_{k=0}^9 \left( -1 \right)^k \cdot \frac{\binom{9}{k}}{2^k} \cdot x^k. 
\]
Here, the fourth term corresponds to $x^3$, that is, $\left( -1 \right)^3 \cdot \binom{9}{3}/2^3 x^3 = - 84/8 x^3 = -21/2 x^3 = -10.5 x^3$. 
The coefficient of $x^5$ is $\left( -1 \right)^5 \cdot \binom{9}{5}/2^5 = - 126/32  = -63/16 = -3.9375$. 

\item[\ref{ex:alternatingsum}]
If $n$ is odd, then every binomial coefficient occurs twice in the sum: once with positive sign, and once with negative sign. 
Indeed, the sign of $\binom{n}{k}$ is $(-1)^k$ and the sign of $\binom{n}{n-k}$ is $(-1)^{n-k}$. 
They cannot have the same sign, because their product is $(-1)^k \cdot (-1)^{n-k} = (-1)^n = -1$, as $n$ is odd. 
Thus every binomial coefficient occurs twice with different signs, their sum is 0, and the whole sum is 0, as well. 

\item[\ref{ex:alternatingsum2}]
Consider the alternating sum of the $n$th row (for $n\geq 1$), 
and use the generating rule of Pascal's triangle: 
{\small
\begin{align*}
& \binom{n}{0} - \binom{n}{1} + \binom{n}{2} - \dots + (-1)^{n-1} \cdot \binom{n}{n-1} + (-1)^{n} \cdot \binom{n}{n} \\
&= \binom{n-1}{0} - \left( \binom{n-1}{0} + \binom{n-1}{1} \right) + \left( \binom{n-1}{1} + \binom{n-1}{2} \right)  - \dots \\
%\left( \binom{n}{2} + \binom{n}{3} \right) 
%+ \dots \\
&+ %(-1)^{n-1} \cdot \left( \binom{n}{n-2} + \binom{n}{n-1} \right)  + 
(-1)^{n-1} \cdot \left( \binom{n-1}{n-2} + \binom{n-1}{n-1} \right) + (-1)^{n} \cdot \binom{n-1}{n-1} \\
%\notag &= \binom{n}{0} + 2 \cdot \left[ \binom{n}{1} + \binom{n}{2} + \binom{n}{3} + \dots + \binom{n}{n-2} + \binom{n}{n-1} \right] + \binom{n}{n} \\
&= \left( \binom{n-1}{0} - \binom{n-1}{0} \right) + \left( - \binom{n-1}{1} + \binom{n-1}{1} \right) + \\
&+\left( \binom{n-1}{2} - \binom{n-1}{2} \right) + \dots 
%&+ \left( \left( -1 \right)^{n-2} \cdot \binom{n}{n-2} + \left( -1 \right)^{n-1} \cdot \binom{n}{n-2} \right)  \\
+ \left( \left( -1 \right)^{n-2} \cdot \binom{n-1}{n-2} + \left( -1 \right)^{n-1} \cdot \binom{n-1}{n-2} \right)  \\
&+ \left( \left( -1 \right)^{n-1} \cdot \binom{n-1}{n-1} + \left( -1 \right)^{n} \cdot \binom{n-1}{n-1} \right) \\
&= 0 + 0 + 0 + \dots + 0 + 0 = 0. 
\end{align*}}
First, we replaced $\binom{n}{0}=1$ by $\binom{n-1}{0}=1$, 
and $\binom{n}{n} = 1$ by $\binom{n-1}{n-1} = 1$, then we used the generating rule of Pascal's triangle. 
Then we observed that every $\binom{n-1}{k}$ occurs twice in the sum: first with a positive sign, then right after it with a negative sign 
(for $0\leq k\leq n-1$). 

\item[\ref{ex:n+mchoosek}]
Again, we give a combinatorial meaning to both sides of \eqref{eq:n+mchoosek}. 
The right hand side gives a clue: 
$\binom{n+m}{l}$ is the number of ways to choose $l$ elements out of an $(n+m)$-element set, 
say 
\[
S = \halmaz{1, 2, \dots , n, n+1, n+2, \dots , n+m}.
\] 
Our plan is to prove that the left hand side of \eqref{eq:n+mchoosek} is the number of $l$-element subsets of $S$, as well. 
Let $S_1 = \halmaz{1, 2, \dots , n}$ and $S_2 = \halmaz{n+1, n+2, \dots , n+m}$. 
Now, try to count the number of ways to choose $l$ elements of $S$ by counting how many elements we choose from $S_1$ and from $S_2$. 
If we choose 0 element from $S_1$, then we must choose $l$ elements from $S_2$. 
We can do this in $\binom{n}{0} \cdot \binom{m}{l}$-many ways. 
If we choose 1 element from $S_1$, then we must choose $l-1$ elements from $S_2$. 
We can do this in $\binom{n}{1} \cdot \binom{m}{l-1}$-many ways. 
If we choose 2 elements from $S_1$, then we must choose $l-2$ elements from $S_2$. 
We can do this in $\binom{n}{2} \cdot \binom{m}{l-2}$-many ways. 
In general, if we choose k elements from $S_1$, then we must choose $l-k$ elements from $S_2$. 
We can do this in $\binom{n}{k} \cdot \binom{m}{l-k}$-many ways. 
In the end, if we choose $l$ elements from $S_1$, then we must choose $0$ element from $S_2$. 
We can do this in $\binom{n}{l} \cdot \binom{m}{0}$-many ways. 
Thus, choosing $l$ elements out of $n+m$ can be done in the following number of ways: 
\[
\binom{n}{0} \cdot \binom{m}{l} + \binom{n}{1} \cdot \binom{m}{l-1} + \dots %+ \binom{n}{n-1} \cdot \binom{n}{1} 
+ \binom{n}{l}\cdot \binom{m}{0} = \sum_{k=0}^l \binom{n}{k} \cdot \binom{m}{l-k}. 
\]
That is, both sides of \eqref{eq:n+mchoosek} counts the number of ways of choosing $l$ elements out of an $(n+m)$-element set, 
and therefore must be equal. 

If we choose $n=m=l$, 
and use the symmetry of Pascal's triangle $\binom{n}{k} = \binom{n}{n-k}$, 
then we obtain exactly equation \eqref{eq:sumsquaresofrow}. 

\item[\ref{ex:n+mchoosek2}]
Consider $(x+y)^{n+m}$, and expand it using the Binomial theorem: 
\[
(x+y)^{n+m} = \sum_{k=0}^{n+m} \binom{n+m}{k} x^{n+m-k} \cdot y^{k}. 
\]
Then the right hand side of \eqref{eq:n+mchoosek} is the coefficient of the term $x^{n+m-l}y^l$. 
We prove that the left hand side is the coefficient of $x^{n+m-l}y^l$, as well. 
For this, we compute $(x+y)^{n+m}$ by multiplying $(x+y)^n \cdot (x+y)^m$ after expanding both factors using the Binomial theorem: 
\[
(x+y)^{n+m} = (x+y)^n \cdot (x+y)^m = \left( \sum_{k=0}^n \binom{n}{k} x^{n-k}y^k \right) \cdot \left( \sum_{k=0}^m \binom{m}{k} x^{m-k}y^k \right). 
\]
Now, let us compute the coefficient of $x^{n+m-l} y^l$. 
When do we obtain $x^{n+m-l} y^l$ when we multiply $\left( \sum_{k=0}^n \binom{n}{k} x^{n-k}y^k \right)$ by $\left( \sum_{k=0}^m \binom{m}{k} x^{m-k}y^k \right)$? 
%Take for example $x^n$ from the first factor, this must be multiplied by $y^n$ from the second factor to obtain $x^n y^n$. 
%The coefficient of $x^n$ in the first factor is $\binom{n}{0}$, the coefficient of $y^n$ in the second factor is $\binom{n}{n}$, 
%thus this multiplication contributes by $\binom{n}{0} \cdot \binom{n}{n}$ to the coefficient of $x^n y^n $ in $(x+y)^{2n}$. 
%Similarly, the term $x^{n-1}y$ from the first factor, this must be multiplied by $xy^{n-1}$ from the second factor to obtain $x^n y^n$. 
%The coefficient of $x^n$ in the first factor is $\binom{n}{1}$, the coefficient of $y^n$ in the second factor is $\binom{n}{n-1}$, 
%thus this multiplication contributes by $\binom{n}{1} \cdot \binom{n}{n-1}$ to the coefficient of $x^n y^n $ in $(x+y)^{2n}$. 
%In general, 
For some $0\leq k\leq l$ the term $x^{n-k}y^k$ in the first factor must be multiplied by $x^{m-l+k} y^{l-k}$ from the second factor. 
The coefficient of $x^{n-k}y^k$ in the first factor is $\binom{n}{k}$, the coefficient of $x^{m-l+k}y^{l-k}$ in the second factor is $\binom{m}{l-k}$, 
thus this multiplication contributes by $\binom{n}{k} \cdot \binom{m}{l-k}$ to the coefficient of $x^{n+m-l} y^l $ in $(x+y)^{n+m}$. 
That is, the coefficient of $x^{n+m-l} y^l$ in $(x+y)^{n+m}$ is 
\[
\sum_{k=0}^l \binom{n}{k} \cdot \binom{m}{l-k}. 
\]
Moreover, the coefficient of $x^{n+m-l}y^l$ in $(x+y)^{n+m}$ is $\binom{n+m}{l}$, thus the two numbers must be equal, 
which proves \eqref{eq:n+mchoosek}: 
%Applying the symmetry of Pascal's triangle (that is, $\binom{n}{k} = \binom{n}{n-k}$), 
%we obtain \eqref{eq:sumsquaresofrow}:
\[
\sum_{k=0}^n \binom{n}{k} \cdot \binom{m}{l-k} = \binom{n+m}{l}. 
\]

\item[\ref{ex:diagonal}]
We can try to prove the identity by induction on $m$. 
For $m=0$ the identity holds, as the left hand side is $\binom{n}{0} = 1$, 
the right hand side is $\binom{n+1}{0} = 1$, as well. 
Assume that the identity holds for $m-1$, that is, 
\[
\sum_{k=0}^{m-1} \binom{n+k}{k} = \binom{n}{0} + \binom{n+1}{1} + \dots + \binom{n+m-1}{m-1} = \binom{n+m}{m-1}. 
\]
This is the induction hypothesis. 
Now we prove the identity for $m$. 
\begin{align*}
& \sum_{k=0}^{m} \binom{n+k}{k} = \underbrace{\binom{n}{0} + \binom{n+1}{1} + \dots + \binom{n+m-1}{m-1}}_{= \binom{n+m}{m-1}, \text{ by the induction hypothesis}} + \binom{n+m}{m} \\
&= \binom{n+m}{m-1} + \binom{n+m}{m} = \binom{n+m+1}{m}. 
\end{align*}
Here, we first used the induction hypothesis, 
then the generating rule of Pascal's triangle (Proposition~\ref{prop:binomsum}). 

We can spare ourselves an induction proof if we use $\binom{n+k}{k} = \binom{n+k}{n}$. 
That is, 
\begin{align*}
&\sum_{k=0}^{m} \binom{n+k}{k} = \binom{n}{0} + \binom{n+1}{1} + \dots + \binom{n+m}{m} \\ 
&= \binom{n}{n} + \binom{n+1}{n} + \dots + \binom{n+m-1}{n} = \sum_{k=n}^{n+m} \binom{k}{n} \\
&= \binom{n+m+1}{n+1} = \binom{n+m+1}{m} 
\end{align*}
by Proposition~\ref{prop:sumkchoosem}. 

\item[\ref{ex:rowp}]
The $k$th element of row $p$ is 
\[
\binom{p}{k} = \frac{p!}{k!\cdot (p-k)!}. 
\]
This number is an integer number, the nominator is divisible by the prime $p$. 
However, for $1\leq k\leq p-1$, the denominator only contains positive integers strictly less than $p$. 
Thus, when we simplify this fraction for obtaining the resulting integer, 
the factor $p$ in the nominator will not cancel out with anything in the denominator, 
and thus $p$ will divide the integer $\binom{p}{k}$. 

If $p$ is not a prime, 
then this property does not necessarily hold.
For example if $n$ is even then by Exercise~\ref{ex:nmidnchoose2} we have $n \nmid \binom{n}{2}$. 
Furthermore, in the first 12 rows we have 
\begin{align*}
6 &\nmid \binom{6}{3} = 20, \\
&& 8 &\nmid \binom{8}{4} = 70, \\
9 &\nmid \binom{9}{3} = 84, \\
&& && 10 &\nmid \binom{10}{5} = 252, \\
12 &\nmid \binom{12}{3} = 220, & 12 &\nmid \binom{12}{4} = 495. 
\end{align*}




\end{enumerate}

\newpage
\section{Recurrence sequences}
\begin{enumerate}
\item[\ref{seq-ex-1}]
We use the notation as follows
1: the largest disk, 2: the second largest disk, 3: the second smallest disk, 4: the smallest disk.
At the beginning there are 4 disks on peg $A$, it is denoted as $\halmaz{1,2,3,4}$, while peg $B$ and peg $C$
has no disks at all, so we write $\halmaz{}$.
\begin{center}
\begin{tabular}{|c|c|c|c|}
\hline
\# move & peg $A$ & peg $B$ & peg $C$\\
\hline
0 & $\halmaz{ 1, 2, 3, 4 }$ & $\halmaz{}$ & $\halmaz{}$\\
\hline
1 & $\halmaz{ 1, 2, 3 }$ & $\halmaz{ 4 }$ & $\halmaz{}$\\
\hline
2 & $\halmaz{ 1, 2 }$ & $\halmaz{ 4 }$ & $\halmaz{ 3 }$\\
\hline
3 & $\halmaz{ 1, 2 }$ & $\halmaz{}$ & $\halmaz{ 3, 4 }$\\
\hline
4 & $\halmaz{ 1 }$ & $\halmaz{ 2 }$ & $\halmaz{ 3, 4 }$\\
\hline
5 & $\halmaz{ 1, 4 }$ & $\halmaz{ 2 }$ & $\halmaz{ 3 }$\\
\hline
6 & $\halmaz{ 1, 4 }$ & $\halmaz{ 2, 3 }$ & $\halmaz{}$\\
\hline
7 & $\halmaz{ 1 }$ & $\halmaz{ 2, 3, 4 }$ & $\halmaz{}$\\
\hline
8 & $\halmaz{}$ & $\halmaz{ 2, 3, 4 }$ & $\halmaz{ 1 }$\\
\hline
9 & $\halmaz{}$ & $\halmaz{ 2, 3 }$ & $\halmaz{ 1, 4 }$\\
\hline
10 & $\halmaz{ 3 }$ & $\halmaz{ 2 }$ & $\halmaz{ 1, 4 }$\\
\hline
11 & $\halmaz{ 3, 4 }$ & $\halmaz{ 2 }$ & $\halmaz{ 1 }$\\
\hline
12 & $\halmaz{ 3, 4 }$ & $\halmaz{ }$ & $\halmaz{ 1, 2 }$\\
\hline
13 & $\halmaz{ 3 }$ & $\halmaz{ 4 }$ & $\halmaz{ 1, 2 }$\\
\hline
14 & $\halmaz{ }$ & $\halmaz{ 4 }$ & $\halmaz{ 1, 2, 3 }$\\
\hline
15 & $\halmaz{ }$ & $\halmaz{ }$ & $\halmaz{ 1, 2, 3, 4 }$\\
\hline
\end{tabular}
\end{center}

\item[\ref{seq-ex-2}]
The idea is to determine geometric progressions satisfying the same recurrence relation as $a_n$.
Let $g_n$ be a geometric progression with the above mentioned property such that $g_n=g_0r^n$ for some $g_0$ and $r$.
It follows that
$$
r^2=7r-10\Rightarrow r^2-7r+10=0.
$$
Solving the quadratic equation yields that $r=2$ or 5. We obtained two appropriate progressions and we know that linear combinations
of these progressions satisfy exactly the same recurrence. Define $W_n$ as follows
$$
W_n=s\cdot 2^n+t\cdot 5^n.
$$
We try to fix $s$ and $t$ such that $W_0=a_0$ and $W_1=a_1$. We get that
\begin{align*}
&W_0=a_0=0,\\
&W_1=a_1=2.
\end{align*}
Therefore
\begin{align*}
s+t&=0,\\
2s+5t&=2.
\end{align*}
The solution of the above system of equations is $s=-2/3, t=2/3$.
Hence
$$
a_n=W_n=-\frac{2}{3}\cdot 2^n+\frac{2}{3}\cdot 5^n.
$$

\item[\ref{seq-ex-3}]
Let $g_n$ be a geometric progression satisfying the same recurrence relation as $a_n$ such that $g_n=g_0r^n$ for some $g_0$ and $r$.
We have that 
$$
r^2=4r-3\Rightarrow r^2-4r+3=0.
$$
That is, $r\in\halmaz{1,3}$. The linear combination we consider now is
$$
W_n=s\cdot 1^n+t\cdot 3^n=s+t\cdot 3^n.
$$
The additional conditions imply that
\begin{align*}
&W_0=a_0=1,\\
&W_1=a_1=13.
\end{align*}
Therefore
\begin{align*}
s+t&=1,\\
s+3t&=13.
\end{align*}
We get the solution $s=-5,t=6$. The explicit formula for $a_n$ is
$$
-5+6\cdot 3^n.
$$

\item[\ref{seq-ex-4}]
This is an example of an order 3 linear recurrence. We define $g_n=g_0r^n$ for some $g_0$ and $r$, which is a geometric
progression. We assume that it satisfies the same recurrence relation as $a_n$, that is, we obtain
$$
r^3=-2r^2+r+2.
$$
It is a cubic polynomial. We look for integer solutions. If there is an integral root, then it divides 2. Hence the possible integral roots
are $\pm 2,\pm 1$.
\begin{center}
\begin{tabular}{|c|c|}
\hline
$r$ & $r^3+2r^2-r-2$\\
\hline
-2 & 0\\
\hline
-1 & 0\\
\hline
1 & 0\\
\hline
2 & 12\\
\hline
\end{tabular}
\end{center}
The cubic polynomial $r^3+2r^2-r-2$ can be written as $\left(x-1\right) \cdot \left(x+1\right) \cdot \left(x+2\right)$, that is, there are three integral roots.
In this case we have three geometric progressions satisfying the recurrence, therefore the appropriate linear combination
is
$$
W_n=s\cdot (-2)^n+t\cdot (-1)^n+u\cdot 1^n.
$$
The corresponding system of linear equations is
\begin{align*}
s+t+u&=0,\\
-2s-t+u&=1,\\
4s+t+u&=2.
\end{align*}
We subtract the first equation from the third one to get $3s=2$. So we have that $s=2/3$.
We eliminate $s$ from the first two equations
\begin{align*}
t+u&=-\frac{2}{3},\\
-t+u&=\frac{7}{3}.
\end{align*}
It is easy to see that $u=5/6$ and $t=-3/2$.
The explicit formula for $a_n$ is 
\[
\frac23 \cdot (-2)^n-\frac32\cdot (-1)^n+\frac56. 
\]

\item[\ref{seq-ex-5}]
The solution is similar to the previous one, so we only provide some details of the computation.
The cubic polynomial in this case is
$$
r^3-6r^2+11r-6.
$$
\begin{center}
\begin{tabular}{|c|c|}
\hline
$r$ & $r^3-6r^2+11r-6$\\
\hline
$-6$ & $-504$\\
\hline
$-3$ & $-120$\\
\hline
$-2$ & $-60$\\
\hline
$-1$ & $-24$\\
\hline
$ 1$ & $0$\\
\hline
$ 2$ & $0$\\
\hline
$ 3$ & $0$\\
\hline
$ 6$ & $60$\\
\hline
\end{tabular}
\end{center}
The three roots of the equation are 1, 2 and 3. Let $W_n=s\cdot 1^n+t\cdot 2^n+u\cdot 3^n$.
The initial values should be equal as well, hence
\begin{align*}
s+t+u&=0,\\
s+2t+3u&=0,\\
s+4t+9u&=1.
\end{align*}
It is easy to eliminate $s$ from the second and the third equation
\begin{align*}
t+2u&=0,\\
3t+8u&=1.
\end{align*}
Therefore $u=1/2, t=-1$ and $s=1/2$. We obtain the following explicit formula
$$
a_n=\frac{1}{2}-2^n+\frac{1}{2}\cdot 3^n.
$$

\item[\ref{seq-ex-6}]
In this exercise we have an order 2 recurrence sequence. Following the method we described 
we get a quadratic polynomial
$$
r^2-4r+4=(r-2)^2.
$$
It has only a multiple root, so we have to consider the linear combinations of $2^n$ and $n\cdot 2^n$.
That is, 
$$
W_n=s\cdot 2^n+tn\cdot 2^n.
$$
The initial values of $a_n$ are $a_0=-1$ and $a_1=0$, hence the system of linear equations is
\begin{align*}
s&=-1,\\
2s+2t&=0.
\end{align*}
It is clear that $s=-1$ and $t=1$. Thus the closed-form formula for $a_n$ is
$$
-2^n+n\cdot 2^n.
$$

\item[\ref{seq-ex-7}]
As before we reduce the problem to a polynomial equation, which is
$$
r^3-5r^2+3r+9.
$$
There are 6 possible integral roots, the divisors of 9, that is, $\halmaz{\pm 9,\pm 3,\pm 1}$.
\begin{center}
\begin{tabular}{|c|c|}
\hline
$r$ & $r^3-5r^2+3r+9$\\
\hline
$-9$ & $-1152$\\
\hline
$-3$ & $-72$\\
\hline
$-1$ & $0$\\
\hline
$1$ & $8$\\
\hline
$3$ & $0$\\
\hline
$9$ & $360$\\
\hline
\end{tabular}
\end{center}
We obtained only 2 roots of the cubic polynomial. Dividing the polynomial $r^3-5r^2+3r+9$ by $(r+1)\cdot (r-3)$
we get $r-3$ and the remainder is 0. That means that 
$$
r^3-5r^2+3r+9=(r+1)\cdot (r-3)^2.
$$
There is a double root, so $W_n$ is defined as
$$
s\cdot (-1)^n+t\cdot 3^n+un\cdot 3^n.
$$
Substituting $n=0,1,2$ yields
\begin{align*}
s+t&=3,\\
-s+3t+3u&=4,\\
s+9t+18u&=29.
\end{align*}
Using that $s=3-t$ we obtain
\begin{align*}
4t+3u&=7,\\
8t+18u&=26.
\end{align*}
So we have the solution $s=2,t=1$ and $u=1$. The explicit formula for $a_n$ is given by
$$
2\cdot (-1)^n+3^n+n\cdot 3^n.
$$

\item[\ref{seq-ex-8}]
Consider the sequence
$$
u_n=\left(\frac{5-3\sqrt{5}}{2}\right)^n+\left(\frac{5+3\sqrt{5}}{2}\right)^n,\quad n\geq 0.
$$
We have $u_0=2$ and $u_1=5$. We try to find a second order linear recurrence satisfied by $u_n$.
If there is such a recurrence, then the corresponding quadratic polynomial is
$$
\left(r-\frac{5-3\sqrt{5}}{2}\right)\cdot \left(r-\frac{5+3\sqrt{5}}{2}\right)=r^2-5r-5.
$$
Therefore the possible recurrence relation is
$$
u_n=5u_{n-1}+5u_{n-2}=5\cdot (u_{n-1}+u_{n-2}).
$$
Now we have that $u_n$ is an integer for all $n$ and $u_n$ is a multiple of 5 if $n\geq 1$.

\item[\ref{seq-ex-9}]
We have a sequence
$$
u_n=(4-\sqrt{3})^n+(4+\sqrt{3})^n,\quad n\geq 0.
$$
One computes that $u_0=2$ and $u_1=8$, that is, the first two elements of the sequence is divisible by 2.
The quadratic polynomial
$$
\left(r-4+\sqrt{2}\right)\cdot \left(r-4-\sqrt{2}\right)=r^2-8r+14
$$
is the polynomial corresponding to the appropriate recurrence relation. Hence the recurrence sequence is given by
\begin{align*}
u_0&=2,\\
u_1&=8,\\
u_n&=8u_{n-1}-14u_{n-2}\quad n\geq 2.
\end{align*}
The statement follows easily since $u_0=2\cdot 1, u_1=2\cdot 4$ and $u_n=2\cdot(4u_{n-1}-7u_{n-2})$.
\end{enumerate}
%\end{enumerate}

\renewcommand{\theenumi}{5.\arabic{enumi}}

